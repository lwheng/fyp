\chapter{Related Work}
\label{Related Works}

%\section{Citation Analysis}
Several authors had researched on works related to Citation Analysis. These works could be categorised into several directions for development. One of them that has a major impact is Citation Classification or similarly, the classification of Citation Function. It aims to determine why the authors of a paper would cite the work of another, and thus better aid readers understand the key ideas presented in a paper. The reasons why authors would cite, are what was meant by the citation function. In an updated version of their paper, Teufel et al. presented an annotation scheme for citation functions \cite{teufel2009annotation}. In this scheme, citations are generalised into 4 main categories: Weak, Contrast, Postive \& Neutral. Some of these categories are further broken down into more specific sub-categories, producing a total of 12 classes for annotating citations. \cite{teufel2006automatic} previously worked on the automatic classification of citation function, utilising features extracted from the \textit{citing context}. \cite{dongensemble} presented an approach to citation classification that uses a combination of various supervised learning algorithms. Similarly, authors worked on analysing the sentiment of citations to determine the polarity of these citations. \cite{athar2011sentiment} used sentence structure based features extracted from the citing context and produced promising results.

In \cite{citation-sensitive} and \cite{csibs}, Wan and his teams built a research tool that acts as a reading aid for readers when browsing through scientific papers. Wan et al. investigated the \textit{literature browsing task} through surveys on researchers who read scientific papers frequently to update themselves. In the initial study conducted by Wan et al., several key ideas were revealed. First, when researchers read scientific papers and see citations made by the author, their main concern, as time-constrained professionals, is whether the cited paper would be worth their time and effort, and money, to follow up on and at the same time, whether to believe in the citation. Second, readers faced the difficulty of finding the exact text that justify the citation. Third, the surveys revealed that readers found it useful if a reading tool could identify important sentences and key words in the cited paper. This study conducted by Wan et al. is based on the fundamental idea of improving the reading experience of practitioners and researchers. The goal is to save a reader's time by helping the reader make relevance judgements about the cited documents. As it is often that readers have to read up on the cited documents to gain a better insight on the current context, this task would be of relevance. The authors then developed the CSIBS based on their studies. The CSIBS tool helps reader determine whether to read on the cited papers by providing a contextual summary of the cited papers.

%\section{Sentence Alignment}
Aligning sentences belonging to similar documents of the same language is an important research area for tasks related to summarisation and paraphrasing. \cite{nelken2006towards} presented a novel algorithm for sentence alignment in monolingual corpora. They showed their approach, which is based on TF*IDF similarity score, produced great precision at aligning sentence, with precision score of 83.1\%. A more recent work by \cite{li2010fast} introduced a new sentence alignment algorithm called Fast-Champollion. Briefly, it splits the input text into alignment fragments and identifies the components of these fragments before aligning them using a Champollion-based algorithm.

Authors paraphrase the content they were referring to usually for greater clarity and to introduce variety. While \cite{shinyama2002automatic} presented an approach to acquire paraphrase automatically, in Citation Provenance we are, in a way, trying to achieve the opposite. By comparing the words and phrases used in a citation with paraphrases extracted from a cited work, one could possibly achieve better sentence alignment between the 2 documents.