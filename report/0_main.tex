\documentclass[hyp, 12pt]{socreport}
\usepackage{fullpage}

\usepackage{url}
\usepackage{amsmath,amsthm,amsfonts}
\usepackage{algorithm,algorithmic}
\usepackage[dvips]{color}
\usepackage{algorithm,algorithmic}
\usepackage{graphicx}
\usepackage{CJKutf8}
\usepackage{multicol}
\setlength{\columnseprule}{0.5pt}
\setlength{\columnsep}{20pt}

\usepackage{tikz}
\usetikzlibrary{trees}
\usetikzlibrary[positioning]
\usetikzlibrary{calc}
\usetikzlibrary{decorations.pathmorphing}
\usetikzlibrary{fit}
\usetikzlibrary{backgrounds}
\tikzstyle{level 1}=[level distance=4cm, sibling distance=3.5cm]
\tikzstyle{level 2}=[level distance=4cm, sibling distance=2cm]
\tikzstyle{bag} = [text width=4em, text centered]
\tikzstyle{end} = [circle, minimum width=3pt,fill, inner sep=0pt]
\tikzstyle{fork} = [circle, minimum width=3pt,fill, inner sep=0pt]

\begin{document}
\pagenumbering{roman}
\title{Citation Provenance}
\author{Heng Low Wee \\ (U096901R)}
\projyear{2011/12}
\projnumber{H079820}
\advisor{A/P Min-Yen Kan}
\deliverables{
	\item Report: 1 Volume
	\item Source Code: 1 DVD}
\maketitle
\begin{abstract}
\paragraph{}
We investigate a new task in citation analysis, {\it citation provenance}, to locate the location of the cited information that justifies a citation. We describe the challenges in collecting annotations for our training set, and present a two-tier approach in tackling this problem. We adopted some features used in Citation Classification and Information Retrieval tasks and with them we differentiate citations that referred to the whole paper in general ({\it general}) versus ones the cited specific claims, evidence or parts of the paper ({\it specific}) and then given that a citation is {\it specific}, locate the cited information in the cited paper. The first tier ({\it GvS}) obtained a promising accuracy of $0.786$ in cross-validation evaluation and in terms of $F_1$ score the second tier ({\it LocateProv}), at $0.90$, performed about $25\%$ better than the baseline.

\begin{descriptors}
	\item H. INFORMATION SYSTEMS
\end{descriptors}
\begin{keywords}
	citation analysis, citation provenance, source of citation, citation classification
\end{keywords}
\begin{implement}
\begin{flushleft}
\hspace{5 mm}Software: Python, NLTK\footnote{http://nltk.org/}, scikit-learn\footnote{http://scikit-learn.org/} \nocite{scikit-learn}\\
\hspace{5 mm}Hardware: MacBook Pro, Intel Core 2 Duo 2.4GHz, 4GB Memory.
\end{flushleft}
\end{implement}
\end{abstract}

\begin{acknowledgement}
\paragraph{}
I would like to express my gratitude to all the volunteer participants from the NUS WING group for participating in my pilot annotation tests. I thank them for testing my annotation scheme, and appreciate the feedback that improved my project.

\paragraph{}
Congrats to the \url{CodeForScience} team from WING NUS, consisting of Jin Zhao, Tao Chen, Eric Yulianto Ang and myself for having emerged the winners for the competition with our prototype application CitWeb that integrated our projects -- Citation Classification and Citation Provenance.

\paragraph{}
Million thanks to Jin Zhao, Tao Chen, and especially my supervisor to this project, A/P Min Yen Kan for providing their guidance during the duration of the project.
\end{acknowledgement}

\listoffigures
\listoftables
\tableofcontents

\chapter{Introduction}
\label{introduction}
% Min: don't need these
% \paragraph{}
Citing previously published scientific papers is an important practice among researchers. It gives credit and acknowledgement to original ideas, and to researchers who did significant work in enabling the current research.  More importantly, it upholds intellectual property. A reader of such research papers often encounters these citations made by the authors in various sentences throughout the paper. Often enough, if a reader wishes to gain a better understanding of the current context, it is necessary to follow these citations and read the cited papers to understand the basis for the current work.  Often, when reading the claims of a sentence supported by a citation, readers wish to know where in the cited paper the information comes from.  

% Min: prefer I and me for thesis.  It's singly authored work.
However, as frequent readers might find, most citations are only \textit{mentions}. They do not directly refer to some particular section of the cited paper, for example, to make reference to the evaluation results made by the authors of the cited paper. Instead, they are what I term {\it general} citations. Other citations refer specifically to particular claims, parts or sections of a paper.  These citations are equally important.  However, since it may not be immediately clear where the cited information is from\footnote{page numbers or references to specific artifacts, such as sections or equation numbers sometimes help to localize such references, but are not often included.}, a reader has to invest additional effort to locate the cited information. We refer to \cite{citation-sensitive} for their survey results to justify our claims. In the series of surveys they conducted, most of their participants found it difficult \textit{finding the exact text to justify the citation}. We quote one of their participants' response directly: ``\textit{Citation usually does not include the position of the information} in the cited article... it might be necessary to read all of the article to find it in another reference and so on." \cite{citation-sensitive}

% Min: Thu Nov  1 09:57:41 SGT 2012 Stopped here
% \paragraph{}
In general, we wish to improve the reading experience of scientific and research documents, from the various fields of research. Readers will be informed of where exactly the cited information is from in the cited paper. We aim to be able to identify which section or paragraph in the referenced paper is the source of the information referred to in the citation, i.e. Citation Provenance. In comparison with \cite{csibs}, which only provided a summarisation solution, we are the first to provide a solution to the difficulty in locating the information that justifies a citation. After highlighting this problem, we hope this would encourage meaning discussions to designing a new citation style that better captures the provenance of the cited information.
 
% \paragraph{}
In the rest of this paper, we will first look at some past works that are related to what we are describing. In Chapter \ref{approach}, we analyse the problem, and describe our approach on tackling the problem. We present our experimental results in Chapter \ref{evaluation} and finally in the end, we conclude our paper together with some further discussion.

\chapter{Related Work}
\label{Related Works}

Citation analysis is broad field of study, which has recently attracted computational methodology, using natural language and machine learning techniques for automation.  We categorise such recent past works into several directions for development.  A subfield of study that has a major impact is citation classification (similarly named as citation function). Such work aims to determine the basis for the authors' citation of the others' work, and thus better aid readers understand the key ideas presented in the paper. The reasons why authors would cite, are what was meant by the citation function. \outcite{teufel2006bannotation} defined an annotation scheme (see Figure~\ref{fig:teufelannotationscheme}) for citation function that is able to describe the relationships between documents linked via citations.

\begin{figure}[h]
\framebox[\textwidth]{
	\begin{tabular}{ l p{11cm}}
		\textsc{Category} & \textsc{Description}\\
		\hline
		Weak & Weakness of cited approach \\
		\hline
		CoCoGM & Contrast/Comparison in Goals or Methods (neutral) \\
		CoCoR0 & Contrast/Comparison in Results (neutral)\\
		CoCo- & Unfavourable Contrast/Comparison (current work is better than cited work)\\
		CoCoXY & Contrast between 2 cited methods \\
		\hline
		PBas & author uses cited work as starting point \\
		PUse & author uses tools/algorithms/data \\
		PModi & author adapts or modifies tools/algorithms/data \\
		PMot & this citation is positive about approach or problem addressed (used to motivate work in current paper) \\
		PSim & author's work and cited work are similar \\
		PSup & author's work and cited work are compatible/provide support for each other \\
		\hline
		Neut & Neutral description of cited work, or not enough textual evidence for above categories or unlisted citation function
	\end{tabular}
}
\caption{12-class annotation scheme designed by \protect\outcite{teufel2006bannotation}}
\label{fig:teufelannotationscheme}
\end{figure}

\outcite{nakov2004citances} discussed the potential of using text surrounding citations, \textit{citances}, for automated analysis of bioscience literature.
%\outcite{teufel2006automatic} previously worked on the automatic classification of citation function, utilising features extracted from the
%\textit{citing context}. 
% Min: we need more detail about this work.  Features?  Classifiers?  Interesting observations?
\outcite{dongensemble} presented an approach to citation classification in which, they extracted several features from \textit{citances}. Some features worth mentioning are their \textit{physical features}, that included the number of unique references cited within the \textit{citances}, and one that measured the existence of cue words. \outcite{teufel2006automatic} also described a similar feature that involved cue phrases, a strong indicator for citation function. Together, these previous works demonstrated the importance of utilising \textit{citances} in citation analysis tasks.

%Similarly, authors worked on analysing the 
%% Min: define sentiment and polarity.  Won't be understandable to those not in NLP.
%sentiment of citations to determine the polarity of these citations. Most recently, \outcite{athar2011sentiment} used sentence structure based features extracted from the citing context to produce 
%% Min: give exact numbers and details on the datasets.
%promising results.

In \cite{citation-sensitive} and \cite{csibs}, Wan and his team built a research tool that acts as a reading aid for readers when browsing through scientific papers. \outcite{csibs} investigated the \textit{literature browsing task} through surveys on researchers who read scientific papers frequently to keep up-to-date themselves. In the initial study conducted by Wan {\it et al.}, several key ideas were revealed. First, when researchers read scientific papers and see citations made by the author, their main concern -- as time-constrained professionals -- is whether the cited paper is worth their effort to follow up on. At the same time, the researchers need to know whether to believe the claim made in the citation. Second, readers faced the difficulty of finding the exact text that justify the citation. Third, the surveys revealed that readers thought that it would be  useful if a reading tool could identify important sentences and key words in the cited paper. This study conducted by \outcite{csibs} is based on the fundamental idea of improving the reading experience of researchers. The goal was to save a reader's time by assisting in the relevance judgement process on the cited documents. As it is often that readers do need to read cited documents to gain insight on the current paper's context, this task is of relevance and importance. The authors then developed the {\it Context Sensitive In-Browser Summariser} (CSIBS) tool based on their studies. Figure~\ref{fig:wanscreenshot} is an overlay of the CSIBS that displays citation-sensitive previews of the cited document. While it highlights matching keywords related to the citation, these sentences on the overlay do not necessarily justify the citation. To locate the provenance solely by word overlap would prove to be ineffective as paraphrasing and re-organising of sentence structure are common when authors cite previous works. There is a need to consider aligning \textit{citances} to the cited document.

\begin{figure}[h]
  \centering
  \includegraphics[scale=0.50]{./wanscreenshot}
  \caption{In-browser overlay preview of the CSIBS}
  \label{fig:wanscreenshot}
\end{figure}

% Min: you need a transition.  I didn't anticipate the topic change.  Why are you talking about paraphrasing and alignment (they are relevant, but why)?
Aligning sentences belonging to similar documents is an important research area for tasks related to summarisation and paraphrasing. \outcite{nelken2006towards} presented a novel algorithm for sentence alignment in for texts in a single language (i.e., monolingual corpora). They showed their approach, which is based on
% Min: need to briefly explain TF.IDF
TF$\times$IDF (a weighting scheme that reflects the importance of a word to a document in a collection of documents) similarity score, produced a high precision (83.1\%) for the task of aligning sentence.
%More recent work by \outcite{li2010fast} introduced a new sentence alignment algorithm called Fast-Champollion. Briefly, it splits the input text into alignment fragments and identifies the components of these fragments before aligning them using a
%% Min: this isn't clear.  Spell out or introduce the method at the high-level in 1-2 sentences.  Why is this related to your work?
%Champollion-based algorithm.
Adding to what we mentioned early, authors paraphrase the content they were referring to usually for greater clarity and to introduce variety. While \outcite{shinyama2002automatic} presented an approach to acquire paraphrase automatically, in our citation provenance project, we aim for the converse goal. By comparing the words and phrases used in a citation with paraphrases extracted from a cited work, one may achieve improved sentence alignment between the two documents.
\chapter{Problem Analysis}
\label{problemanalysis}

For the scope of our project, we define all citations as belonging to one of two types: \textbf{General} and \textbf{Specific}. 
% Min: didn't quite get this sentence.  What does it mean?
We define citations as such to be inline with our goal. That is, given a citation as Specific, that there is a specific portion of the cited document that contains the original information to support the citation. Otherwise, the citation is General. We use the following guideline to ensure that there is no ambiguity in our definition of the dichotomy between General and Specific: \\ \\
\textbf{General Citations}
\begin{enumerate}
\item Authors may refer to a paper as a whole. If the author cites the cited paper for a key idea, e.g. Machine Learning, and Machine Learning makes up the entire or majority of the cited paper, it is a general citation.
% Min: do you want to be more specific about "contributions"?
\item Authors may refer to a paper as a form of mentioning. In such cases, the authors merely mention the cited paper to acknowledge its contributions.
\end{enumerate}
\textbf{Specific Citations}
\begin{enumerate}
\item Authors may refer to a term definition in the cited paper.
\item Authors may refer to a key idea/implementation in the cited paper. This key idea or implementation does not constitute the entire or majority of the cited paper.
\item Authors may refer to an algorithm or a theorem in the cited paper. This algorithm/theorem and its supplementary evidence does not constitute the entire or majority of the cited paper.
\item Authors may refer to particular digits or numerical figures in the cited paper. This is usually done to make reference to quantitative evaluation results in the cited paper. Authors may also complement the cited paper for its performance.
\item Authors may quote a passage in the cited paper.
\end{enumerate}

\begin{figure}[h]
\label{fig:terminology}
\framebox[\textwidth]{
	\begin{tabular}{ l p{11cm}}
		\textsc{Term} & \textsc{Description}\\
		\hline
% Min: need to define "in-line citation"?
		Citing Paper & The paper that makes the citation \\
		Cited Paper & The paper that is being cited by the citing paper \\
% Min: you need to make sure that the format Xdd-dddd is known to the reader as standing for a paper.
% Min: if possible, show these relationships also in a graphical figure.  You can use powerpoint or something to draw and export as a .eps or .pdf
		Cite Link & E.g. \url{E06-1034==>J93-2004}. A citation relation between a citing paper (\url{E06-1034}) and a cited paper (\url{J93-2004}) \\
		Cite String & The citation mark. E.g. Nivre and Scholz (2004), [1], (23) \\
		Citing Sentence & A sentence in the citing paper that contains the in-line citation. E.g. \textit{That algorithm, in turn, is similar to the dependency parsing algorithm of \textbf{Nivre and Scholz (2004)}, but it builds a constituent tree and a dependency tree simultaneously.} \\
		Citing Context & The block of text surrounding the citing sentence, about 2 sentences before and after the citing sentence, for providing contextual information \\
		Cited Fragment & A fragment, from a few lines to paragraphs, in the cited paper
	\end{tabular}
}
\caption{Terminological conventions used in this dissertation}
\end{figure}

% Min: not clear.  Is this the goal, what is already done or what is to be done?
For \textbf{Specific} citations, we specifically extract a fragment in the cited paper that represents the source of the information mentioned in the citation itself, i.e. citation provenance.

\section{Scope Of The Problem}
We decompose the task into to parts.  In the first part, we first determine whether a citation is General or Specific. If a citation is General, the reader can be directed, for example, to the abstract of the cited paper. If a citation is Specific, the reader can be directed to the specific paragraph or line.  From our definition, given that a citation is Specific, then there must exists a region in the cited paper that the citation refers to. To solve this second part, we implement a ranking system that determines the location of this region.

Our project has a practical aspect that can be readily applied to scholarly papers.  However, even with the components above, to field the project practically, the system must be able to detect in-line citations in a suitable textual representation of a scholarly paper corpus. 
While such engineering issues are important, our work focuses only on the research aspects of determining citation provenance, and hence we abstracted away the problem of locating in-line citations.  Instead we reduced the problem to only determining the type of a citation and its location. To solve the practical problem of locating the in-line citations, we utilize the open-source logical document structure and reference string parsing system, ParsCit, developed in \outcite{parscit}. Conveniently, ParsCit identifies the citing sentence, together with its citing context.

\section{Modelling The Problem As Search}
In web search engines, an user enters a search query, and a search engine would use this query to search within its search domain -- millions of web pages -- and then display the best matching web pages as compared to the search query. 
% Min: unclear.  What do you mean "having a search query"?
That would be equivalent to having a search query for an entire corpus of research papers. Our second subproblem of linking Specific citations to their origin can be modelled as search.

Consider reading a paper, \url{A}. We know the citations made by \url{A}, and these cited papers are listed in its References section. From this our search domain for any query from \url{A} would be the contents of the list of cited papers. We reduce this search domain further when we are investigating a particular citation in \url{A}, say now paper \url{A} cites the paper \url{B}. Now, for this citation, the scope of search would be the sub-domain -- contents of paper \url{B}. So instead of searching for the best matching document in the corpus, we are now searching within \url{B}. The search query is the citation from \url{A}, and the {\it candidate documents} are the various regions (fragments) in \url{B} (Refer to Figure \ref{fig:model} for a simple illustration). With the help of ParsCit \cite{parscit}, the citing context can be extracted. The search query is then citing context which consists of the citing sentence.

\begin{figure}[h]
  \centering
  \includegraphics[scale=0.50]{./model}
  \caption{Modeling Our Problem}
  \label{fig:model}
\end{figure}

Cast this way, our first subproblem is simply a \textit{binary classification problem}, where we attempt to determine whether a fragment is either General or Specific.

\section{Target Corpus}
\label{buildingcorpus}
We selected the ACL Anthology Reference Corpus\footnote{http://acl-arc.comp.nus.edu.sg/} (ACL-ARC) as the target corpus to perform our research on. The ACL-ARC consists of publications covering topics in computational linguistics.  While we wish to generalize our citation provenance methodology to work on publications from all fields of research, we chose to start with this corpus as it provides the \textit{interlink data} that conveniently informs us of the cite links between the papers in the corpus. For instance, in the interlink data, a link like \url{X98-103==>X96-1049} says that the paper \url{X98-103}\footnote{All ACL-ARC papers are assigned an unique paper ID} cites \url{X96-1049}.  

% Min: the $n$ factor is usually very small so it doesn't seem O(nm).  Rather instead of the $n$ citing contexts it is that there are $k$ references.  The complexity is then mostly dominated by O(km)
Given our formal problem statement, we can now specify the required data for the task. For each cite link, there can be multiple in-line citations i.e. multiple citing contexts, when a target paper is cited multiple times by the authors. Each citing context is compared with every fragment in the cited paper; i.e., if a cite link has $n$ citing contexts and the cited paper can be divided into $m$ fragments.  This immediately gives rise to $(n \times m)$ data instances.
% Min: show some examples of data instances.

% Min: you need to give some reasoning about why annotations are needed before proceeding into the annotations themselves.
% Min: GLOBAL you still tend to use passive tense and modals.  Try to use active and simple tenses.
% Min: GLOBAL use \url.  I've fixed one URL below.  Don't use URL for non-URLs because in PDFs the URLs will be active hyperlinks (that then won't work)
\subsection*{Collecting Annotations -- First Attempt}
The first attempt at collecting annotations was to require an annotator to specify the line numbers of the cited information that the citing context was referring to. The annotator would be provided the citing and cited paper in plain text format, and he/she will need to annotate on a separate file, specifying the line number range, e.g. line range \url{L12-55} of the cited paper. For this annotation task, we designed an annotation framework\footnote{\url{http://citprov.heroku.com}} where an annotator is presented with an user-friendly interface to select the lines in the cited paper that he/she deem Specific. We posted this task onto the Amazon Mechanical Turk (MTurk\footnote{https://www.mturk.com}) as an attempt to collect annotations on a larger scale and we collected some annotations from a few MTurk workers. After a trial round of annotation, we reviewed this annotation scheme together with feedback from the small group of participants.

% Min: I recall you did an IRB application.  You should mention that, especially if it was approved to be exempted.  That is a significant amount of effort.
First, this annotation task is non-trivial. Participants must be able to understand the contents of the papers, and thus, must largely be either subject matter experts (researchers) or have some experience in reading scientific papers. While it is possible to target a selected category of MTurk workers for this task, the complexity of this task requires participants with research experiences, which could be limited in numbers. Furthermore, most of the annotations collected from MTurk do not agree among the annotators and ourselves. Thus we abandoned collecting annotations via MTurk, and performed annotations manually.

Second, this annotation scheme is too tricky, and would also cause us much problem when it comes to evaluation. Consider an implemented system that outputs a prediction for citation provenance in the form of a line number range. It is difficult to judge the correctness of this prediction, say \url{L50-78}, when compared against the annotated \url{L12-55}. The prediction \textit{overlaps} the annotation by 5 lines, but this variable amount of overlap is not definitive and difficult to decide at what extent of overlap only do we consider the prediction correct. Thus we switched to the alternative.

\subsection*{Collection Annotations -- Second Attempt}
The second attempt is more straightforward. Recall that we used ParsCit for extracting the citing context. ParsCit also divides a paper into logically adequate fragments according to sections, sub-sections, figures and tables etc. So instead of annotating the papers in plain text format by line number ranges, we annotated the structured output from ParsCit, each of the fragments of the cited papers with 3 classes: General ($g$), Specific-Yes ($y$) and Specific-No ($n$). To be precise, we annotated $g$ (for all its fragments) if a cite link is deemed General, and $y$ \underline{only} for the fragment(s) that is deemed Specific. For the other fragments that are not Specific, we annotated $n$. Table \ref{tab:annotation} summarises the statistics for annotation. Note that only percentage values for Specific instances are displayed.

\begin{table}[h]
	\center
	\begin{tabular}{ l | l}
		\textsc{Item} & \textsc{Statistics}\\
		\hline
		No. of Cite Links & 275 (7.6\% Specific) \\
		No. of Fragments & 30943 (0.09\% Specific-Yes, 12.9\% Specific-No)
	\end{tabular}
	\caption{Annotation Statistics}
	\label{tab:annotation}
\end{table}

Specific citations are very rare and the training data is heavily skewed towards General citations. After prolonged periods of searching for valid Specific citations in our training corpus, we argue that despite more attempts to gather more positive instances, the ratio between General and Specific would remain the same. This challenging situation we have with the annotations also contributes to our approach to the problem, as we explain in the following chapter.

During the annotation process, we observed that Specific citations can be categorised into four sub-classes.  We acknowledge that these observations are limited to the particular corpus we worked with and may not generalize.  We observed that Specific citations may:
\begin{enumerate}
\item refer to digits/numerical figures in the cited paper, usually in the evaluation section
\item refer to term definitions by the author(s) of the cited paper
\item refer to algorithms/theorems in the cited paper
\item quote a line or segment in the cited paper
\end{enumerate}
These observations also led to the implementation of some features that are defined next chapter in our approach.

\chapter{Approach}
\label{approach}
I propose a two-tier approach to our problem. In the first tier, it plays the role of a \textit{filter}, and attempts to filter out the General citations, leaving behind the Specific citations to be passed to the second tier. Figure \ref{fig:twotier} illustrates the flow of our approach.
\begin{figure}[h]
  \centering
  \includegraphics[scale=0.60]{./twotier}
  \caption{A Two-Tier Approach}
  \label{fig:twotier}
\end{figure}

\section{{\it GvS} (First Tier)}
\label{firsttier}
{\it GvS}, short for General versus Specific, is the first tier in my approach to filter out the General citations. In {\it GvS}, we are performing a 2-class \textit{citation classification} task, which already is a challenging task in the research area of citation analysis. We are not interested in determining whether the citation is one of the 12 class as defined by \cite{teufel2009annotation}, but only whether it is General or Specific. {\it GvS} makes use of information only from the citing contexts in a citing paper. I built a model based on features extracted from the citing contexts. With this model, {\it GvS} classifies citing contexts into one of the two classes. Only those contexts that are classified as Specific will be passed to the second tier.

\subsection*{Building The Model For {\it GvS}}
To build a model to classify General versus Specific, we adopt some of the features that \cite{dongensemble} used for citation classification. From each of the 275 annotated cite links mentioned in Table \ref{tab:annotation} I extract a set of features into a {\it feature vector} and map it to its {\it label} according to annotation (Figure \ref{fig:featurevector}). The features used are as described below.

\begin{figure}[h]
\centering
$v_1:[f_1, f_2, f_3, \ldots, f_n] \rightarrow L_1$ \\
$v_2:[f_1, f_2, f_3, \ldots, f_n] \rightarrow L_2$ \\
$\vdots$ \\
$v_i:[f_1, f_2, f_3, \ldots, f_n] \rightarrow L_i$ \\
$\vdots$ \\
$v_m:[f_1, f_2, f_3, \ldots, f_n] \rightarrow L_m$
\caption{Mapping feature vectors to labels from annotation}
\label{fig:featurevector}
\end{figure}

\subsection*{{\it GvS} Features}
\begin{enumerate}
\item Physical Features (Feature $A$)\\
We adopted the physical features as presented in \cite{dongensemble}. They are:
\begin{enumerate}
\item \textit{Location}: in which section the citing sentence is from.
\item \textit{Popularity}: no. of citation marks in the citing sentence.
\item \textit{Density}: no. of unique citation marks in the citing sentence and its neighbour sentences.
\item \textit{AvgDens}: the average of Density among the citing and neighbour sentences.
\end{enumerate}
The intuition I have for using this feature is: A higher number of citation marks within a citing sentence suggests these citations are likely to be General since there was little room of discussion by the author(s).

\item Number Density (Feature $B$)\\
A numerical feature similar to the first feature set that measures the density of numerical figures in the citing context. The intuition is that Specific citations tend refer numerical figures in evaluation results in the cited paper. E.g. ``...Nivre and Scholz (2004) obtained a precision of 79.1\%...". This feature was added based on the observations I made earlier in Chapter \ref{buildingcorpus}.

\item Publishing Year Difference (Feature $C$)\\
A numerical feature that represents difference in the publishing year between the citing and cited paper. The intuition is that higher difference suggests cited paper is older and presented a fundamental idea, and thus cited for General purposes.

\item Citing Context's Average \url{tf-idf} Weight (Feature $D$)\\
A numerical feature that indicates the average amount of \textit{valuable} words, as determined by \url{tf-idf} \cite{irtextbook},  in the citing context. Higher values suggest important words and thus specific keywords.

\item Cue Words (Feature $E$)\\
Another numerical feature adopted from \cite{dongensemble}, that computes the amount of cue words that appear in the citing sentence and its neighbour sentences. We defined 2 classes of cue words: Cue-General and Cue-Specific (refer to Appendix \ref{cuewords} for list of cue words). These cue words are hand-picked based on the examples I observed during the annotation process.
\end{enumerate}

Recall that according to our annotation statistics, this task is heavily skewed towards General citations. Building a model based on this skewed set of data instances will only produce a bias model that often predicts General. In fact, during some preliminary experiments where all data instances are fitted into the model, it outputs General for all its predictions. To fix this problem, I propose training the model on {\it unskewed data}.

From the set of labelled feature vectors, I first gathered the Specific instances. Then I {\bf randomly} selected from the rest to have a $1:1$ of Specific vs General instances. While this ratio appear unrealistic compared to the actual statistics, I argue that I am building a model using balanced data to measure its ability to differentiate between the 2 types of citation.

\section{{\it LocateProv} (Second Tier)}
\label{secondtier}
{\it LocateProv}, short for Locate Provenance, is the second tier of my approach. The design of {\it LocateProv} is all its inputs are Specific citations predicts which of the fragments in the cited paper is the cited fragment. Resembling a search, in {\it LocateProv} the citing context becomes the {\it query} to match the cited fragments in the cited paper. For that I also added some features that are common in Information Retrieval tasks.

\subsection*{Building The Model For {\it LocateProv}}
In {\it LocateProv} we are predicting which cited fragment is the provenance of a citation. Instead of cite links, I used the annotated fragments in Table \ref{tab:annotation} to build the model. In this tier the features used are based on both the citing contexts and the cited fragments in order to {\it connect} the citation to its provenance. Similarly the feature vectors are mapped onto the annotated labels.

\subsection*{{\it LocateProv} Features}
\begin{enumerate}
\item Surface Matching (Feature $F$)\\
A numerical feature that measures the amount of word overlap between the citing sentence and a fragment in the cited paper.

\item Number Near-Miss (Feature $G$)\\
A numerical feature that measures the amount of numerical figures overlap between the citing sentence and a fragment in the cited paper. This feature will preprocess each fragment, rounding numerical figures or converting to percentage values when it tries to match the numerical figures in the citing sentence. This feature was added because of the observations I made earlier in Chapter \ref{buildingcorpus}, that citations may refer to evaluation results in the cited paper.

\item Bigrams Matching (Feature $H$)\\
A numerical feature that measures the amount of bigrams overlap between the citing sentence and a fragment in the cited paper. This feature was added to preserve word order when comparing the citing sentence and the fragment. This feature was also targeted at Specific citations that refer to term definitions or quote directly.

\item Cosine Similarity (Feature $I$)\\
A common numerical feature used in Information Retrieval tasks to measure similarity between the query and a candidate document. In our case, citing sentence and the fragment.
\end{enumerate}
Most of these features are added based on some of the observations I made during the annotation tasks.

Again, recall that the data instances that were annotated are heavily skewed against Specific citations. In fact, the ratio of Specific-Yes instances compared to the rest is at least $1:1000$. It is impossible to train a model that is not bias with this entire set of instances. Hence I used the same method used in {\it GvS}: to use a $1:1$ of Specific-Yes versus Specific-No instances. Note that this also coincide with the design of {\it LocateProv} that inputs are only Specific citations. It was also not feasible to use the actual ratio between Specific-Yes and Specific-No because comparing a citing-cited pair of papers, the ratio of citing context to the number of fragments in the cited paper is easily $1:100$.

For both tiers, I trained the models using various classifiers and evaluated their performances on a few evaluation strategies. I discuss the evaluation process in the following chapter.
\chapter{Evaluation}
\label{evaluation}
\paragraph{}
We performed 2 evaluations, one for each tier as described early in Chapter \ref{twotierapproach}. We are able to do this because the tiers are independent of each other.

%Tao: To make the results more complete, you may discuss the discriminative power of features by show the weights learned by machine learning algorithem. 
\section{Results - First Tier}
\paragraph{}
Recall that we have 275 annotated cite links, either General ($g$) or Specific ($s$), and that we have very limited instances of Specific cite links, a situation mentioned in \cite{li2010negative}, that we have a highly unbalanced ratio between General instances and Specific instances. So for our evaluation, we first gathered all Specific instances, and then \textbf{randomly} select General instances. Out of these 56 instances, we have $1:1$ ratio of Specific versus General instances. The reason for choosing a $1:1$ ratio is because we wish to measure our approach's ability to differentiate between General and Specific when given a balanced training set.

\paragraph{}
We trained our model using various classifiers, and then performed \url{Leave-One-Out} (\url{LOO}) and \url{n-Fold} ($n=14$, each fold has 4 instances) evaluation using the 56 instances. The main reason for using \url{Leave-One-Out} is because we are working with limited instances and we wish to maximise them for training. To test our method's performance when given less training data, we use the \url{n-Fold} strategy. When trained on SVM, we observed that in terms of \textit{feature weights}, feature 1(a) and 5 were assigned the highest weights. For instance, feature 1(a), \textit{Density}, was assigned with a weight of magnitude 0.703. This suggests the amount of in-line citations and the amount of cue words within the citing context play an important part in our prediction task.

We evaluated using various classifiers and Table \ref{tab:firsttieresults} summarises the performance for each classifier.

\begin{table}[h]
	\center
	\begin{tabular}{ c | c  c  c }
		\textsc{Classifier/Values} & \textsc{Avg. Precision} & \textsc{Avg. Recall} & \textsc{Avg. F$_1$-Score} \\
		& \url{LOO} / \url{n-Fold} & \url{LOO} / \url{n-Fold} & \url{LOO} / \url{n-Fold} \\
		\hline
		\textsc{SVM} 			& 0.84 / 0.71 & 0.84 / 0.70 & 0.84 / 0.69 \\
		\textsc{NaiveBayes} 	& 0.70 / 0.70 & 0.66 / 0.66 & 0.64 / 0.64 \\
		\textsc{DecisionTree}	& 0.72 / 0.66 & 0.71 / 0.66 & 0.71 / 0.66
	\end{tabular}
	\caption{First Tier Results}
	\label{tab:firsttieresults}
\end{table}

\paragraph{}
We examine the confusion matrix for the best performing SVM classifier that we ran for the \url{Leave-One-Out} strategy. Recall that our First Tier's objective is to filter out the General citations. Our goal is to attain higher numbers in both the $g$-$g$ and $s$-$s$ cells in the confusion matrix. We achieved this in Table \ref{tab:svmconfusionmatrix} and we can conclude that our First Tier performed well in differentiating General and Specific citations.

\begin{table}[h]
	\center
	\begin{tabular}{ c | c  c }
		 & \textsc{actual $g$} & \textsc{actual $s$} \\
		\hline
		\textsc{predicted $g$} 	& 24 & 4 \\
		\textsc{predicted $s$}		& 5 & 23
	\end{tabular}
	\caption{Confusion Matrix for SVM with Leave-One-Out}
	\label{tab:svmconfusionmatrix}
\end{table}

\section{Results - Second Tier}
\paragraph{}
For Second Tier evaluation, we are predicting whether each fragment in the cited paper is a Specific one. We have over 30 thousand training instances for second tier, and so for the same reason, we had to select our training set manually similarly to get a $1:1$ ratio for Specific versus General instances.

\begin{table}[h]
	\center
	\begin{tabular}{ c | c  c  c }
		\textsc{Classifier/Values} & \textsc{Avg. Precision} & \textsc{Avg. Recall} & \textsc{Avg. F$_1$-Score} \\
		& \url{LOO} / \url{n-Fold} & \url{LOO} / \url{n-Fold} & \url{LOO} / \url{n-Fold} \\
		\hline
		\textsc{SVM} 			& 0.85 / 0.84 & 0.84 / 0.82 & 0.84 / 0.82 \\
		\textsc{NaiveBayes} 	& 0.80 / 0.78 & 0.79 / 0.77 & 0.78 / 0.77 \\
		\textsc{DecisionTree}	& 0.89 / 0.86 & 0.89 / 0.86 & 0.89 / 0.86
	\end{tabular}
	\caption{Second Tier Results}
	\label{tab:secondtieresults}
\end{table}
\newpage
\paragraph{}
In this case, the Decision Tree classifier performed slightly better than SVM. Similarly, our goal is to attain higher numbers in both the $g$-$g$ and $s$-$s$ cells in the confusion matrix and in Table \ref{tab:decisiontreeconfusionmatrix} we achieved good results. This means, given a Specific citation, this approach would perform well in determining whether a fragment in the cited paper is the cited fragment or not.

\begin{table}[h]
	\center
	\begin{tabular}{ c | c  c }
		 & \textsc{actual $g$} & \textsc{actual $s$} \\
		\hline
		\textsc{predicted $g$} 	& 23 & 5 \\
		\textsc{predicted $s$}		& 3 & 25
	\end{tabular}
	\caption{Confusion Matrix for Naive Bayes}
	\label{tab:decisiontreeconfusionmatrix}
\end{table}

(Refer to Appendix \ref{resultsdetails} for more details of our experimental results.)
\chapter{Discussion}
\label{discussion}
% Min: You need to actually discuss performance of your classifier somewhere here too.  It doesn't make sense to have an evaluation section talking about percentages and a discussion section that doesn't follow up.
Citation Provenance is a task that has had little development up to now. In this thesis, we have rigorously defined the problem and decomposed it into two tiers.  A key limitation in enabling this research is the difficulty in obtaining Specific citations in scientific papers in the ground truth. 
% Min: you need to mention what neutral citations are for.  Some people will just read the conclusion and introduction, so this may not make sense.
\outcite{teufel2009annotation} showed that the percentage of neutral citations was 62.7\%. 
% Min: I really don't understand what you are saying here.  
% Min: Also, did you end up using Eric's classifier as an input?  This sentence seems to be related to that.
We can say that the percentage of General citations is at least as much, because our definition of a Specific citation is more restricted compared to the 12 classes defined by \outcite{teufel2009annotation}. This supports our observations during annotation collection that most citations are mere {\it mentions}.

Sometimes, in-line citations to scientific papers in journals and books capture the chapter numbers and page numbers. The main reason is because the length of the cited document is very long compared to the citing document. An example of such citation is \textit{(J. Doe, 2012, sec. 6.5, 174-85)}. In this citation it captures the section number, ``\textit{sec. 6.5}", and page numbers, ``\textit{174-85}", to a book or journal. Note that the granularity of such style is not specific enough for our problem as a section can be arbitrary lengthy. In our thesis, we have used an archive of computational linguistics papers due to their convenient citation linking data.  However, they are usually less than 20 pages, which is much shorter than books and journals.  While outside of the scope of this thesis, here we sketch a new citation style that may better captures citation provenance.

Our sketched style is straightforward: Authors should numerically label each segment or fragment in the cited paper. This applies to text bodies, figures and tables. An example for a Specific citation: \textit{(B. White, 2011, \textbf{B23})}. Notice we added another a \textbf{B} to \textbf{23}, which could be a better way to distinguish between text bodies (B), figures (F) and tables (T). \textbf{23} simply means the 23rd segment of the type B. Suppose the cited paper is already labelled, when a reader sees a citation a paper, the reader sees there is the additional information at the end of the citation and understands it is a Specific citation. To read up on the cited paper would be a breeze.

We argue that even though the occurrence of Specific citations is low in the examined dataset, citation provenance is an important reading tool to assist in the understanding and navigation between papers linked by citations. 
% Min: not grammatical.  
% Min: you should factor this into a separate subsection, putting in description and screenshots of your CFS application. 
% Min: don't use url for non-urls.  BUG.  Please fix.
We support with evidence the validity of our claim, that a prototype application (that performs Citation Provenance) submitted as part of the \url{Code For Science}\footnote{http://www.codeforscience.com/singapore} 2012 competition organised by Elsevier was well received among the judging panel that consisted of professionals from fields related to information technology and libraries.


\chapter{Conclusion}
\label{conclusion}
We touched on a new task for citation analysis, Citation Provenance. In this task, we are trying to locate the information in a cited paper that justifies a citation found in a citing paper.

I presented a two-tier approach towards this problem, {\it Gvs} and {\it LocateProv}. With the first acting as a filter to separate the General citations from the Specific ones and the second one to predict which of the fragments in the cited paper are referenced by the citation. One of the challenges in this task is the highly unbalanced ratio between General versus Specific citations. Also, the annotation task is very challenging and would require experienced researchers who understands the content of the papers to be annotated. As a result all the training instances were manually annotated.

To train prediction models for this task, I gathered an unskewed set of instances, a balanced ratio of General versus Specific instances, and measured their ability to differentiate between the 2 types of citations. Feature analysis showed that most of the features are essential, with the Physical Features (Feature $A$) adopted from \cite{dongensemble} proving to have the most discriminative power in {\it GvS}, and Cosine Similarity (a common strategy for Information Retrieval tasks) remained to be most important in {\it LocateProv}.

Finally, evaluations on {\it Gvs} and {\it LocateProv} produced promising results in classifying General versus Specific citations and locating the cited fragment in the cited paper. 

\bibliographystyle{socreport}
\bibliography{bibsource}
\appendix
\chapter{Cue Words}
\label{cuewords}
The following is the list of cue words used in one of our feature. During feature extraction, all words are stemmed before we make any comparison.
\section{Cue-General}
proposed, propose, presented, present, suggested, suggests, described, describe, discuss, discussed, gave, introduction, introduced, shown, showed, sketched, sketch, talked, adopted, adopt, based, originated, originate, built, researchers, comparative, comparison, following, previously, previous

\section{Cue-Specific}
obtains, obtained, score, scored, high, F-score, Precision, precision, Recall, recall, estimated, estimates, reported, reports, probability, probabilities, peaked, experimental, experimented, rate, error

\chapter{Results Details (1st Tier)}
\label{resultsdetails}
\section{Results: Leave-One-Out}
\begin{table}[ht]
\begin{minipage}[b]{0.45\linewidth}\centering
\begin{tabular}{ c | c  c  c }
	& \textsc{Precision} & \textsc{Recall} & \textsc{F$_1$-Score} \\
	\hline
	\textsc{$g$} 	& 0.83 & 0.86 & 0.84 \\
	\textsc{$s$}	& 0.85 & 0.82 & 0.84
\end{tabular}
\end{minipage}
\hspace{0.5cm}
\begin{minipage}[b]{0.45\linewidth}
\centering
\begin{tabular}{ c | c  c }
	 & \textsc{actual $g$} & \textsc{actual $s$} \\
	\hline
	\textsc{predicted $g$} 	& 24 & 4 \\
	\textsc{predicted $s$}		& 5 & 23
\end{tabular}
\end{minipage}
\caption{SVM $P/R/F_1$ Scores and Confusion Matrix}
\end{table}

\begin{table}[ht]
\begin{minipage}[b]{0.45\linewidth}\centering
\begin{tabular}{ c | c  c  c }
	& \textsc{Precision} & \textsc{Recall} & \textsc{F$_1$-Score} \\
	\hline
	\textsc{$g$} 	& 0.61 & 0.89 & 0.72 \\
	\textsc{$s$}	& 0.80 & 0.43 & 0.56
\end{tabular}
\end{minipage}
\hspace{0.5cm}
\begin{minipage}[b]{0.45\linewidth}
\centering
\begin{tabular}{ c | c  c }
	 & \textsc{actual $g$} & \textsc{actual $s$} \\
	\hline
	\textsc{predicted $g$} 	& 25 & 3 \\
	\textsc{predicted $s$}		& 16 & 12
\end{tabular}
\end{minipage}
\caption{Naive Bayes $P/R/F_1$ Scores and Confusion Matrix}
\end{table}

\begin{table}[ht]
\begin{minipage}[b]{0.45\linewidth}\centering
\begin{tabular}{ c | c  c  c }
	& \textsc{Precision} & \textsc{Recall} & \textsc{F$_1$-Score} \\
	\hline
	\textsc{$g$} 	& 0.70 & 0.75 & 0.72 \\
	\textsc{$s$}	& 0.73 & 0.68 & 0.70
\end{tabular}
\end{minipage}
\hspace{0.5cm}
\begin{minipage}[b]{0.45\linewidth}
\centering
\begin{tabular}{ c | c  c }
	 & \textsc{actual $g$} & \textsc{actual $s$} \\
	\hline
	\textsc{predicted $g$} 	& 21 & 7 \\
	\textsc{predicted $s$}		& 9 & 19
\end{tabular}
\end{minipage}
\caption{Decision Tree $P/R/F_1$ Scores and Confusion Matrix}
\end{table}
\newpage

\section{Results: n-Fold}
\begin{table}[ht]
\begin{minipage}[b]{0.45\linewidth}\centering
\begin{tabular}{ c | c  c  c }
	& \textsc{Precision} & \textsc{Recall} & \textsc{F$_1$-Score} \\
	\hline
	\textsc{$g$} 	& 0.76 & 0.57 & 0.65 \\
	\textsc{$s$}	& 0.66 & 0.82 & 0.73
\end{tabular}
\end{minipage}
\hspace{0.5cm}
\begin{minipage}[b]{0.45\linewidth}
\centering
\begin{tabular}{ c | c  c }
	 & \textsc{actual $g$} & \textsc{actual $s$} \\
	\hline
	\textsc{predicted $g$} 	& 16 & 12 \\
	\textsc{predicted $s$}		& 5 & 23
\end{tabular}
\end{minipage}
\caption{SVM $P/R/F_1$ Scores and Confusion Matrix}
\end{table}

\begin{table}[ht]
\begin{minipage}[b]{0.45\linewidth}\centering
\begin{tabular}{ c | c  c  c }
	& \textsc{Precision} & \textsc{Recall} & \textsc{F$_1$-Score} \\
	\hline
	\textsc{$g$} 	& 0.61 & 0.89 & 0.72 \\
	\textsc{$s$}	& 0.80 & 0.43 & 0.56
\end{tabular}
\end{minipage}
\hspace{0.5cm}
\begin{minipage}[b]{0.45\linewidth}
\centering
\begin{tabular}{ c | c  c }
	 & \textsc{actual $g$} & \textsc{actual $s$} \\
	\hline
	\textsc{predicted $g$} 	& 25 & 3 \\
	\textsc{predicted $s$}		& 16 & 12
\end{tabular}
\end{minipage}
\caption{Naive Bayes $P/R/F_1$ Scores and Confusion Matrix}
\end{table}

\begin{table}[ht]
\begin{minipage}[b]{0.45\linewidth}\centering
\begin{tabular}{ c | c  c  c }
	& \textsc{Precision} & \textsc{Recall} & \textsc{F$_1$-Score} \\
	\hline
	\textsc{$g$} 	& 0.66 & 0.68 & 0.67 \\
	\textsc{$s$}	& 0.67 & 0.64 & 0.65
\end{tabular}
\end{minipage}
\hspace{0.5cm}
\begin{minipage}[b]{0.45\linewidth}
\centering
\begin{tabular}{ c | c  c }
	 & \textsc{actual $g$} & \textsc{actual $s$} \\
	\hline
	\textsc{predicted $g$} 	& 19 & 9 \\
	\textsc{predicted $s$}		& 10 & 18
\end{tabular}
\end{minipage}
\caption{Decision Tree $P/R/F_1$ Scores and Confusion Matrix}
\end{table}



\chapter{Results Details (2nd Tier)}
\section{Results: Leave-One-Out}
\begin{table}[ht]
\begin{minipage}[b]{0.45\linewidth}\centering
\begin{tabular}{ c | c  c  c }
	& \textsc{Precision} & \textsc{Recall} & \textsc{F$_1$-Score} \\
	\hline
	\textsc{$g$} 	& 0.91 & 0.75 & 0.82 \\
	\textsc{$s$}	& 0.79 & 0.93 & 0.85
\end{tabular}
\end{minipage}
\hspace{0.5cm}
\begin{minipage}[b]{0.45\linewidth}
\centering
\begin{tabular}{ c | c  c }
	 & \textsc{actual $g$} & \textsc{actual $s$} \\
	\hline
	\textsc{predicted $g$} 	& 21 & 7 \\
	\textsc{predicted $s$}		& 2 & 26
\end{tabular}
\end{minipage}
\caption{SVM $P/R/F_1$ Scores and Confusion Matrix}
\end{table}

\begin{table}[ht]
\begin{minipage}[b]{0.45\linewidth}\centering
\begin{tabular}{ c | c  c  c }
	& \textsc{Precision} & \textsc{Recall} & \textsc{F$_1$-Score} \\
	\hline
	\textsc{$g$} 	& 0.74 & 0.89 & 0.81 \\
	\textsc{$s$}	& 0.86 & 0.68 & 0.76
\end{tabular}
\end{minipage}
\hspace{0.5cm}
\begin{minipage}[b]{0.45\linewidth}
\centering
\begin{tabular}{ c | c  c }
	 & \textsc{actual $g$} & \textsc{actual $s$} \\
	\hline
	\textsc{predicted $g$} 	& 25 & 3 \\
	\textsc{predicted $s$}		& 9 & 19
\end{tabular}
\end{minipage}
\caption{Naive Bayes $P/R/F_1$ Scores and Confusion Matrix}
\end{table}

\begin{table}[ht]
\begin{minipage}[b]{0.45\linewidth}\centering
\begin{tabular}{ c | c  c  c }
	& \textsc{Precision} & \textsc{Recall} & \textsc{F$_1$-Score} \\
	\hline
	\textsc{$g$} 	& 0.92 & 0.86 & 0.89 \\
	\textsc{$s$}	& 0.87 & 0.93 & 0.90
\end{tabular}
\end{minipage}
\hspace{0.5cm}
\begin{minipage}[b]{0.45\linewidth}
\centering
\begin{tabular}{ c | c  c }
	 & \textsc{actual $g$} & \textsc{actual $s$} \\
	\hline
	\textsc{predicted $g$} 	& 24 & 4 \\
	\textsc{predicted $s$}		& 2 & 26
\end{tabular}
\end{minipage}
\caption{Decision Tree $P/R/F_1$ Scores and Confusion Matrix}
\end{table}
\newpage

\section{Results: n-Fold}
\begin{table}[ht]
\begin{minipage}[b]{0.45\linewidth}\centering
\begin{tabular}{ c | c  c  c }
	& \textsc{Precision} & \textsc{Recall} & \textsc{F$_1$-Score} \\
	\hline
	\textsc{$g$} 	& 0.91 & 0.71 & 0.80 \\
	\textsc{$s$}	& 0.76 & 0.93 & 0.84
\end{tabular}
\end{minipage}
\hspace{0.5cm}
\begin{minipage}[b]{0.45\linewidth}
\centering
\begin{tabular}{ c | c  c }
	 & \textsc{actual $g$} & \textsc{actual $s$} \\
	\hline
	\textsc{predicted $g$} 	& 20 & 8 \\
	\textsc{predicted $s$}		& 2 & 26
\end{tabular}
\end{minipage}
\caption{SVM $P/R/F_1$ Scores and Confusion Matrix}
\end{table}

\begin{table}[ht]
\begin{minipage}[b]{0.45\linewidth}\centering
\begin{tabular}{ c | c  c  c }
	& \textsc{Precision} & \textsc{Recall} & \textsc{F$_1$-Score} \\
	\hline
	\textsc{$g$} 	& 0.73 & 0.86 & 0.79 \\
	\textsc{$s$}	& 0.83 & 0.68 & 0.75
\end{tabular}
\end{minipage}
\hspace{0.5cm}
\begin{minipage}[b]{0.45\linewidth}
\centering
\begin{tabular}{ c | c  c }
	 & \textsc{actual $g$} & \textsc{actual $s$} \\
	\hline
	\textsc{predicted $g$} 	& 24 & 4 \\
	\textsc{predicted $s$}		& 9 & 19
\end{tabular}
\end{minipage}
\caption{Naive Bayes $P/R/F_1$ Scores and Confusion Matrix}
\end{table}

\begin{table}[ht]
\begin{minipage}[b]{0.45\linewidth}\centering
\begin{tabular}{ c | c  c  c }
	& \textsc{Precision} & \textsc{Recall} & \textsc{F$_1$-Score} \\
	\hline
	\textsc{$g$} 	& 0.88 & 0.82 & 0.85 \\
	\textsc{$s$}	& 0.83 & 0.89 & 0.86
\end{tabular}
\end{minipage}
\hspace{0.5cm}
\begin{minipage}[b]{0.45\linewidth}
\centering
\begin{tabular}{ c | c  c }
	 & \textsc{actual $g$} & \textsc{actual $s$} \\
	\hline
	\textsc{predicted $g$} 	& 23 & 5 \\
	\textsc{predicted $s$}		& 3 & 25
\end{tabular}
\end{minipage}
\caption{Decision Tree $P/R/F_1$ Scores and Confusion Matrix}
\end{table}

\end{document}
