\documentclass[hyp]{socreport}
\usepackage{fullpage}

\usepackage{url}
\usepackage{amsmath,amsthm,amsfonts}
\usepackage{algorithm,algorithmic}
\usepackage[dvips]{color}
\usepackage{algorithm,algorithmic}
\usepackage{graphicx}
\usepackage{CJKutf8}
\usepackage{multicol}
\setlength{\columnseprule}{0.5pt}
\setlength{\columnsep}{20pt}

\usepackage{tikz}
\usetikzlibrary{trees}
\usetikzlibrary[positioning]
\usetikzlibrary{calc}
\usetikzlibrary{decorations.pathmorphing}
\usetikzlibrary{fit}
\usetikzlibrary{backgrounds}
\tikzstyle{level 1}=[level distance=4cm, sibling distance=3.5cm]
\tikzstyle{level 2}=[level distance=4cm, sibling distance=2cm]
\tikzstyle{bag} = [text width=4em, text centered]
\tikzstyle{end} = [circle, minimum width=3pt,fill, inner sep=0pt]
\tikzstyle{fork} = [circle, minimum width=3pt,fill, inner sep=0pt]

\begin{document}
\pagenumbering{roman}
\title{Citation Provenance}
\author{Heng Low Wee \\ (U096901R)}
\projyear{2011/12}
\projnumber{H079820}
\advisor{A/P Min-Yen Kan}
\deliverables{
	\item Report: 1 Volume
	\item Source Code: 1 DVD}
\maketitle
\begin{abstract}
\paragraph{}
We investigate a new task in citation analysis, Citation Provenance. That is, to determine where in the cited paper is the referenced information for a citation in a citing paper. We describe the challenges in collecting annotations for our training set, and present a two-tier approach in tackling this problem. From our evaluation results, we showed that the features we introduced into our approach were able to differentiate General versus Specific citations, and also able to determine which is the cited fragment in the cited paper.

\begin{descriptors}
	\item TO BE COMPLETED
\end{descriptors}
\begin{keywords}
	citation analysis, citation provenance, source of citation
\end{keywords}
\begin{implement}
\begin{flushleft}
\hspace{5 mm}Software: Python, NLTK, scikit-learn\\
\hspace{5 mm}Hardware: MacBook Pro, Intel Core 2 Duo 2.4GHz, 4GB Memory.
\end{flushleft}
\end{implement}
\end{abstract}

\begin{acknowledgement}
\paragraph{}
I would like to express my gratitude to all the volunteer participants from the NUS WING group for taking part in my two pilot testings on collecting annotations for my project. I thank you for taking your time to try out our annotation scheme, and I appreciate your feedbacks that made my project better.

\paragraph{}
Million thanks to Jin Zhao, Tao Chen, and especially my supervisor to this project, A/P Min Yen Kan for providing their guidance during the duration of the project.
\end{acknowledgement}

\listoffigures
\listoftables
\tableofcontents

\chapter{Introduction}
\label{introduction}
% Min: don't need these
% \paragraph{}
Citing previously published scientific papers is an important practice among researchers. It gives credit and acknowledgement to original ideas, and to researchers who did significant work in enabling the current research.  More importantly, it upholds intellectual property. A reader of such research papers often encounters these citations made by the authors in various sentences throughout the paper. Often enough, if a reader wishes to gain a better understanding of the current context, it is necessary to follow these citations and read the cited papers to understand the basis for the current work.  Often, when reading the claims of a sentence supported by a citation, readers wish to know where in the cited paper the information comes from.  

% Min: prefer I and me for thesis.  It's singly authored work.
However, as frequent readers might find, most citations are only \textit{mentions}. They do not directly refer to some particular section of the cited paper, for example, to make reference to the evaluation results made by the authors of the cited paper. Instead, they are what I term {\it general} citations. Other citations refer specifically to particular claims, parts or sections of a paper.  These citations are equally important.  However, since it may not be immediately clear where the cited information is from\footnote{page numbers or references to specific artifacts, such as sections or equation numbers sometimes help to localize such references, but are not often included.}, a reader has to invest additional effort to locate the cited information. We refer to \cite{citation-sensitive} for their survey results to justify our claims. In the series of surveys they conducted, most of their participants found it difficult \textit{finding the exact text to justify the citation}. We quote one of their participants' response directly: ``\textit{Citation usually does not include the position of the information} in the cited article... it might be necessary to read all of the article to find it in another reference and so on." \cite{citation-sensitive}

% Min: Thu Nov  1 09:57:41 SGT 2012 Stopped here
% \paragraph{}
In general, we wish to improve the reading experience of scientific and research documents, from the various fields of research. Readers will be informed of where exactly the cited information is from in the cited paper. We aim to be able to identify which section or paragraph in the referenced paper is the source of the information referred to in the citation, i.e. Citation Provenance. In comparison with \cite{csibs}, which only provided a summarisation solution, we are the first to provide a solution to the difficulty in locating the information that justifies a citation. After highlighting this problem, we hope this would encourage meaning discussions to designing a new citation style that better captures the provenance of the cited information.
 
% \paragraph{}
In the rest of this paper, we will first look at some past works that are related to what we are describing. In Chapter \ref{approach}, we analyse the problem, and describe our approach on tackling the problem. We present our experimental results in Chapter \ref{evaluation} and finally in the end, we conclude our paper together with some further discussion.

\chapter{Related Work}
\label{Related Works}

Citation analysis is broad field of study, which has recently attracted computational methodology, using natural language and machine learning techniques for automation.  We categorise such recent past works into several directions for development.  A subfield of study that has a major impact is citation classification (similarly named as citation function). Such work aims to determine the basis for the authors' citation of the others' work, and thus better aid readers understand the key ideas presented in the paper. The reasons why authors would cite, are what was meant by the citation function. \outcite{teufel2006bannotation} defined an annotation scheme (see Figure~\ref{fig:teufelannotationscheme}) for citation function that is able to describe the relationships between documents linked via citations.

\begin{figure}[h]
\framebox[\textwidth]{
	\begin{tabular}{ l p{11cm}}
		\textsc{Category} & \textsc{Description}\\
		\hline
		Weak & Weakness of cited approach \\
		\hline
		CoCoGM & Contrast/Comparison in Goals or Methods (neutral) \\
		CoCoR0 & Contrast/Comparison in Results (neutral)\\
		CoCo- & Unfavourable Contrast/Comparison (current work is better than cited work)\\
		CoCoXY & Contrast between 2 cited methods \\
		\hline
		PBas & author uses cited work as starting point \\
		PUse & author uses tools/algorithms/data \\
		PModi & author adapts or modifies tools/algorithms/data \\
		PMot & this citation is positive about approach or problem addressed (used to motivate work in current paper) \\
		PSim & author's work and cited work are similar \\
		PSup & author's work and cited work are compatible/provide support for each other \\
		\hline
		Neut & Neutral description of cited work, or not enough textual evidence for above categories or unlisted citation function
	\end{tabular}
}
\caption{12-class annotation scheme designed by \protect\outcite{teufel2006bannotation}}
\label{fig:teufelannotationscheme}
\end{figure}

\outcite{nakov2004citances} discussed the potential of using text surrounding citations, \textit{citances}, for automated analysis of bioscience literature.
%\outcite{teufel2006automatic} previously worked on the automatic classification of citation function, utilising features extracted from the
%\textit{citing context}. 
% Min: we need more detail about this work.  Features?  Classifiers?  Interesting observations?
\outcite{dongensemble} presented an approach to citation classification in which, they extracted several features from \textit{citances}. Some features worth mentioning are their \textit{physical features}, that included the number of unique references cited within the \textit{citances}, and one that measured the existence of cue words. \outcite{teufel2006automatic} also described a similar feature that involved cue phrases, a strong indicator for citation function. Together, these previous works demonstrated the importance of utilising \textit{citances} in citation analysis tasks.

%Similarly, authors worked on analysing the 
%% Min: define sentiment and polarity.  Won't be understandable to those not in NLP.
%sentiment of citations to determine the polarity of these citations. Most recently, \outcite{athar2011sentiment} used sentence structure based features extracted from the citing context to produce 
%% Min: give exact numbers and details on the datasets.
%promising results.

In \cite{citation-sensitive} and \cite{csibs}, Wan and his team built a research tool that acts as a reading aid for readers when browsing through scientific papers. \outcite{csibs} investigated the \textit{literature browsing task} through surveys on researchers who read scientific papers frequently to keep up-to-date themselves. In the initial study conducted by Wan {\it et al.}, several key ideas were revealed. First, when researchers read scientific papers and see citations made by the author, their main concern -- as time-constrained professionals -- is whether the cited paper is worth their effort to follow up on. At the same time, the researchers need to know whether to believe the claim made in the citation. Second, readers faced the difficulty of finding the exact text that justify the citation. Third, the surveys revealed that readers thought that it would be  useful if a reading tool could identify important sentences and key words in the cited paper. This study conducted by \outcite{csibs} is based on the fundamental idea of improving the reading experience of researchers. The goal was to save a reader's time by assisting in the relevance judgement process on the cited documents. As it is often that readers do need to read cited documents to gain insight on the current paper's context, this task is of relevance and importance. The authors then developed the {\it Context Sensitive In-Browser Summariser} (CSIBS) tool based on their studies. Figure~\ref{fig:wanscreenshot} is an overlay of the CSIBS that displays citation-sensitive previews of the cited document. While it highlights matching keywords related to the citation, these sentences on the overlay do not necessarily justify the citation. To locate the provenance solely by word overlap would prove to be ineffective as paraphrasing and re-organising of sentence structure are common when authors cite previous works. There is a need to consider aligning \textit{citances} to the cited document.

\begin{figure}[h]
  \centering
  \includegraphics[scale=0.50]{./wanscreenshot}
  \caption{In-browser overlay preview of the CSIBS}
  \label{fig:wanscreenshot}
\end{figure}

% Min: you need a transition.  I didn't anticipate the topic change.  Why are you talking about paraphrasing and alignment (they are relevant, but why)?
Aligning sentences belonging to similar documents is an important research area for tasks related to summarisation and paraphrasing. \outcite{nelken2006towards} presented a novel algorithm for sentence alignment in for texts in a single language (i.e., monolingual corpora). They showed their approach, which is based on
% Min: need to briefly explain TF.IDF
TF$\times$IDF (a weighting scheme that reflects the importance of a word to a document in a collection of documents) similarity score, produced a high precision (83.1\%) for the task of aligning sentence.
%More recent work by \outcite{li2010fast} introduced a new sentence alignment algorithm called Fast-Champollion. Briefly, it splits the input text into alignment fragments and identifies the components of these fragments before aligning them using a
%% Min: this isn't clear.  Spell out or introduce the method at the high-level in 1-2 sentences.  Why is this related to your work?
%Champollion-based algorithm.
Adding to what we mentioned early, authors paraphrase the content they were referring to usually for greater clarity and to introduce variety. While \outcite{shinyama2002automatic} presented an approach to acquire paraphrase automatically, in our citation provenance project, we aim for the converse goal. By comparing the words and phrases used in a citation with paraphrases extracted from a cited work, one may achieve improved sentence alignment between the two documents.
\chapter{Our Approach}
\label{approach}
\section{Terminology}
\paragraph{}
To aid the reader, and to avoid misunderstanding and confusion, it is important that we first list some of the key terms we are using in our paper.
\begin{table}[h]
	\center
	\begin{tabular}{ l p{13cm}}
		\textsc{Term} & \textsc{Description}\\
		\hline
		Citing Paper & The paper that makes the citation \\
		Cited Paper & The paper that is being cited by the citing paper \\
		Cite Link & E.g. \url{E06-1034==>J93-2004}. A citation relation between a citing paper (\url{E06-1034}) and a cited paper (\url{J93-2004}) \\
		Cite String & The citation mark. E.g. Nivre and Scholz (2004), [1], (23) \\
		Citing Sentence & A sentence in the citing paper that contains the in-line citation. E.g. \textit{That algorithm, in turn, is similar to the dependency parsing algorithm of \textbf{Nivre and Scholz (2004)}, but it builds a constituent tree and a dependency tree simultaneously.} \\
		Citing Context & The block of text surrounding the citing sentence, usually 2 sentences before and after the citing sentence, for providing contextual information \\
		Cited Fragment & A fragment, from a few lines to paragraphs, in the cited paper
	\end{tabular}
	\caption{Terminology}
	\label{tab:terminology}
\end{table}

\section{Problem Analysis}
\subsection{Types of Citation}
\paragraph{}
In the scope of our project, all citations could be classified into 2 types: \textbf{General}, and \textbf{Specific}. We define citations as such to be inline with our goal. That is, to be able to tell, if specific, where the cited information is in the cited document. Otherwise, the citation would be deemed general. To rid of ambiguity in our definition of a general/specific citation, we have the following guidelines:\\
\textbf{General Citations}
\begin{enumerate}
\item Authors refer to a paper as a whole. If the author cites for a key idea, e.g. Machine Learning, and the entire or majority of the cited paper is about Machine Learning, it is a general citation.
\item Authors refer to a paper as a form of mentioning. The authors merely mentions the cited paper out of acknowledgement of its contributions.
\end{enumerate}
\textbf{Specific Citations}
\begin{enumerate}
\item Authors refer to a term definition in the cited paper.
\item Authors refer to a key idea/implementation in the cited paper. This key idea/implementation does not make up the entire cited paper.
\item Authors refer to an algorithm or a theorem in the cited paper. This algorithm/theorem does not make up the entire cited paper.
\item Authors refer to digits or numerical figures in the cited paper. Usually for making reference to evaluation results in the cited paper.
\item Authors quote a line/segment in the cited paper.
\end{enumerate}
\paragraph{}
In general, for \textbf{Specific} citations, we would be able to specifically extract a fragment in the cited paper that represents the source of the information mentioned in the citation itself i.e. Citation Provenance.

\subsection{Locating The Cited Information}
\paragraph{}
Our problem is now reduced to determining whether a citation is General or Specific. If a citation is general, the reader can be directed, for example, to the Abstract section of the cited paper, but this is not the main focus of our task. If a citation is specific, the reader can be directed to that specific paragraph or lines respectively. Therefore during computation, the cited document can be broken down into fragments. Hence if given that a citation is specific, then there must exists a fragment that the citation refers to. For this we need to implement some ranking system that determines the location of this fragment.

\subsection{Scope Of The Problem}
\paragraph{}
In this project, we abstract away the problem of locating the in-line citations in a paper, and reduce our problem to only determining the type of a citation and its location. To solve the problem of locating the in-line citations, we utilize the open-source ParsCit system developed by \cite{parscit}. Conveniently, ParsCit identifies the citing sentence, together with the citing context.

\subsection{Modelling The Problem As Search}
\paragraph{}
In web search engines, an user enters a search query, and a search engine would use this query to search within its search domain -- millions of web pages -- and then display the best matching web pages as compared to the search query. That would be equivalent to having a search query for an entire corpus of research papers. Our problem can also be modelled as a searching problem, but a reduced version as compared to web search engines.

\paragraph{}
Consider reading a paper, \url{A}. We know the citations made by \url{A}, and these cited papers are listed in the References section of \url{A}. From this our search domain for any query from \url{A} would be the contents of the list of cited papers. We reduce this search domain further when we are investigating a particular citation in \url{A}, say now paper \url{A} cites the paper \url{B}. Now, for this citation, the scope of search would be the sub-domain -- contents of paper \url{B}. So instead of searching for the best matching document in the corpus, we are now searching within \url{B}. Our problem analysis tells that we have to break down \url{B} into fragments, and the search query would be for these fragments (Refer to Figure \ref{fig:model} for a simple illustration). With the help of ParsCit \cite{parscit}, the citing sentence can be extracted. The search query would be citing context which consists of the citing sentence.

\begin{figure}[h]
  \centering
  \includegraphics[scale=0.50]{./model}
  \caption{Modeling Our Problem}
  \label{fig:model}
\end{figure}

\paragraph{}
Our problem is now a \textit{binary classification problem}, where we attempt to determine whether a fragment is either General or Specific.

\section{Training Corpus}
\paragraph{}
At this initial stage, we picked the ACL Anthology Reference Corpus\footnote{http://acl-arc.comp.nus.edu.sg/} (ACL-ARC). The ACL-ARC consists of publications of the Computational Linguistics field. Note that in general, we wish to perform this citation provenance task on all publications from all fields of research. This corpus is chosen as a start, because it provides the \textit{interlink data} that conveniently informs us of the cite links between the papers in the corpus. For instance, in the interlink data, a link like \url{X98-103 ==> X96-1049} says that the paper \url{X98-103} cites \url{X96-1049}.

\subsection{Collecting Annotations}
\paragraph{}
Now that we have modelled our problem, we are able to specify the required data format for our task. For each cite link, there can be multiple in-line citations i.e. multiple citing contexts. For each citing context, we are comparing with each fragment in the cited paper. In other words, if a cite link has $n$ citing contexts and the cited paper can be divided into $m$ fragments, immediately we have $(n \times m)$ instances.

\paragraph{}
Our first attempt at collecting annotations was to require an annotator to specify the line numbers of the cited information that the citing context was referring to. The annotator would be provided the citing and cited paper in plain text format, and he/she will need to annotate on a separate file, specifying the line number range, e.g. line range \url{L12-55} of the cited paper. For this annotation task, we designed an annotation framework\footnote{http://citprov.heroku.com} where an annotator is presented with an user-friendly interface to select the lines in the cited paper that he/she deem Specific. We posted this task onto the Amazon Mechanical Turk (MTurk\footnote{https://www.mturk.com}) for a few Turk workers to participate in our annotation task. After a pilot round of collection, we reviewed this annotation scheme together with feedbacks from our small group of participants. First, this annotation task is a non-trivial one. The annotations collected from MTurk do not agree among the annotators and ourselves. Participants must be able to understand the contents of the papers, thus, must be researchers or have some experience in reading scientific papers. Second, this annotation scheme is too tricky, and would also cause us much problem when it comes to evaluation. Consider our implemented system that outputs a prediction for citation provenance in the form of a line number range. It is difficult to judge the correctness of this prediction, say \url{L30-67}, when compared against the annotated \url{L12-55}.

\paragraph{}
Our second attempt is more straightforward. Recall that we use ParsCit for extracting the citing context. ParsCit also divides a paper into logically adequate fragments according to sections, sub-ssections, figures and tables etc. So instead of annotating by line number ranges, we annotated each fragment with 3 classes: General ($g$), Specific-Yes ($y$) and Specific-No ($n$). To be precise, we annotate $g$ if a cite link is deemed General, and $y$ \underline{only} for the fragment(s) that is deemed Specific. For the other fragments that are not Specific, we annotate $n$. Table \ref{tab:annotation} summarises the statistics for annotation. Note that we only display percentage values for Specific instances.
\begin{table}[h]
	\center
	\begin{tabular}{ l | l}
		\textsc{Item} & \textsc{Statistics}\\
		\hline
		No. of Cite Links & 275 (7.6\% Specific) \\
		No. of Fragments & 30943 (0.09\% Specific-Yes, 12.9\% Specific-No)
	\end{tabular}
	\caption{Annotation Statistics}
	\label{tab:annotation}
\end{table}

\paragraph{}
As one can see, Specific citations are very rare. From a machine learning point of view, immediately one can observe that the training data is skewed towards General citations. From this we may conclude that even if we attempt to gather more positive instances, the ratio between General and Specific should remain about the same. This situation we have with our annotations also contributes to our approach to the problem, as we explain in the following section.

\section{A Two-Tier Approach}
\paragraph{}
We propose a two-tier approach to our problem. In the first tier, it plays the role of a \textit{filter}, and attempts to filter out the General citations, leaving behind the Specific citations to be passed to the second tier. Figure \ref{fig:twotier} illustrates the flow of our approach.
\begin{figure}[h]
  \centering
  \includegraphics[scale=0.60]{./twotier}
  \caption{A Two-Tier Approach}
  \label{fig:twotier}
\end{figure}

\subsection{First Tier}
\paragraph{}
The First Tier is our attempt to filter out the General citations. In this tier, we are performing a 2-class citation classification task. We are not interested in determining whether the citation is one of the 12 class as defined by \cite{teufel2009annotation}, but only whether it is General or Specific. For each cite link we extract its citing contexts. Then for these contexts we extract feature vectors in order to pass it into our prediction model. We do adopt similar features that were presented in previous works on citation classification.

\subsubsection{First Tier Features}
\begin{enumerate}
\item Physical Features \\
We adopted the physical features as presented in \cite{dongensemble}. They are:
\begin{enumerate}
\item \textit{Location}: in which section the citing sentence is from.
\item \textit{Popularity}: no. of references in the citing sentence.
\item \textit{Density}: no. of unique references in the citing sentence and its neighbour sentences.
\item \textit{AvgDens}: the average of Density among the citing and neighbour sentences.
\end{enumerate}

\item Number Density \\
A numerical feature we formulated that computes the density of numerical figures in the citing context. The intuition is that Specific citations refer to evaluation results in the cited paper. E.g. ``...Nivre and Scholz (2004) obtained a precision of 79.1\%...".

\item Publishing Year Difference \\
A numerical feature that represents difference in the publishing year between the citing and cited paper. The intuition is that higher difference suggests cited paper is older and presented a fundamental idea.

\item Citing Context's Average TF-IDF Weight \\
A numerical feature that indicates the amount of \textit{valuable} (as determined by TF-IDF \cite{irtextbook}) words in the citing context. Higher values suggest important words and thus specific keywords.

\item Cue Words \\
Another numerical feature adopted from \cite{dongensemble}, that computes the amount of cue words (pre-defined manually by us) that appear in the citing sentence and its neighbour sentences. We defined 2 classes of cue words: Cue-General and Cue-Specific. Cue-General $ = \{proposed, introduced, sketched, discussed, suggested\ldots \}$ and Cue-Specific $ = \{obtained, scored, precision, probabilities, experimental\ldots \}$. These cue words are selected based on the examples we observed in our training corpus.
\end{enumerate}

From our training corpus we extracted these features to build our First Tier Model for prediction.

\subsection{Second Tier}
\paragraph{}
In our Second Tier, it is another abstraction of our problem. We assume all the inputs into the second tier are Specific citations, and then we attempt to predict which of the fragments in the cited paper is the cited fragment.

\subsubsection{Second Tier Features}
\begin{enumerate}
\item Surface Matching \\
A numerical feature that measures the amount of word overlap between the citing sentence and a fragment in the cited paper.

\item Number Near-Miss \\
A numerical feature that measures the amount of numerical figures overlap between the citing sentence and a fragment in the cited paper. This feature will preprocess each fragment, rounding numerical figures or converting to percentage values, when it try to match the numerical figures in the citing sentence. The intuition for this feature is from our observation that most Specific citations refer to evaluation results in the cited paper.
\end{enumerate}
\chapter{Evaluation}
\label{evaluation}
\paragraph{}
We performed 2 evaluations, one for each tier as described early in Chapter \ref{twotierapproach}. We are able to do this because the tiers are independent of each other.

\section{Results - First Tier}
\paragraph{}
Recall that we have 275 annotated cite links, either General ($g$) or Specific ($s$), and that we have very limited instances of Specific cite links, a situation mentioned in \cite{li2010negative}, that we have a highly unbalanced ratio between General instances and Specific instances. So for our evaluation, we first gathered all Specific instances, and then randomly select General instances, twice the number of Specific instances. Out of these 84 instances, we have $1:2$ ratio of Specific vs General instances.

\paragraph{}
We trained our model using various classifiers, and then performed \url{Leave-One-Out} evaluation using the 84 instances. In other words, in one round of evaluation we predict 84 times. For each classifier, we performed 10 rounds of evaluation and we compute the average score for each round. Table \ref{tab:firsttieresults} summarises the accuracy for each classifier.

\begin{table}[h]
	\center
	\begin{tabular}{| l | l | l |}
		\hline
		\textsc{SVM} & \textsc{NaiveBayes} & \textsc{DecisionTree} \\
		\hline
		
	\end{tabular}
	\caption{Annotation Statistics}
	\label{tab:firsttieresults}
\end{table}
\input{5_conclusion} 
\chapter{Discussion}
\label{discussion}
\paragraph{}
Citation Provenance is a task that has relatively little developments done on it. In our paper we attempt to define the nature of the problem, and presented a possible approach to tackle it. One of the main challenges we had with this task is the limited number of Specific citations in scientific papers. \cite{teufel2009annotation} showed that the percentage of neutral citations was 62.7\%. We can say that the percentage of General citations is at least as much, because our definition of a Specific citation is much more restricted compared to the 12 classes defined in \cite{teufel2009annotation}. We observed, and conclude that in general research \& scientific papers, most citations are neutral and General.

\paragraph{}
We argue, that even though the percentage of Specific citations is very low and that the value of applications that perform such task seems low, citation provenance would prove to be an important reading tool that helps readers understand and navigate between papers that are linked via citations. We support with evidence the validity of our claim, that our prototype application submitted as part of the \url{Code For Science}\footnote{http://www.codeforscience.com/singapore} 2012 competition organised by Elsevier was well received among the judging panel that consisted of professionals from fields related to information technology and libraries.

%Tao: Include the actual link (http://www.codeforscience.com/singapore) of codeforscience competition 

\bibliographystyle{socreport}
\bibliography{bibsource}
\appendix
\chapter{Cue Words}
\label{cuewords}
The following is the list of cue words used in one of our feature. During feature extraction, all words are stemmed before we make any comparison.
\section{Cue-General}
proposed, propose, presented, present, suggested, suggests, described, describe, discuss, discussed, gave, introduction, introduced, shown, showed, sketched, sketch, talked, adopted, adopt, based, originated, originate, built, researchers, comparative, comparison, following, previously, previous

\section{Cue-Specific}
obtains, obtained, score, scored, high, F-score, Precision, precision, Recall, recall, estimated, estimates, reported, reports, probability, probabilities, peaked, experimental, experimented, rate, error

\chapter{Results Details}
\label{resultsdetails}

\begin{table}[h]
	\center
	\begin{tabular}{ c | c  c  c }
		& \textsc{Precision} & \textsc{Recall} & \textsc{F1-Score} \\
		\hline
		\textsc{$g$} 	& 0.77 & 0.71 & 0.74 \\
		\textsc{$s$}	& 0.73 & 0.79 & 0.76
	\end{tabular}
	\caption{First Tier: SVM P/R/F scores}
\end{table}

\begin{table}[h]
	\center
	\begin{tabular}{ c | c  c }
		 & \textsc{actual $g$} & \textsc{actual $s$} \\
		\hline
		\textsc{predicted $g$} 	& 20 & 8 \\
		\textsc{predicted $s$}		& 6 & 22
	\end{tabular}
	\caption{First Tier: Confusion Matrix for SVM}
\end{table}

\begin{table}[h]
	\center
	\begin{tabular}{ c | c  c  c }
		& \textsc{Precision} & \textsc{Recall} & \textsc{F1-Score} \\
		\hline
		\textsc{$g$} 	& 0.60 & 0.86 & 0.71 \\
		\textsc{$s$}	& 0.75 & 0.43 & 0.55
	\end{tabular}
	\caption{First Tier: Naive Bayes P/R/F scores}
\end{table}

\begin{table}[h]
	\center
	\begin{tabular}{ c | c  c }
		 & \textsc{actual $g$} & \textsc{actual $s$} \\
		\hline
		\textsc{predicted $g$} 	& 24 & 4 \\
		\textsc{predicted $s$}		& 16 & 12
	\end{tabular}
	\caption{First Tier: Confusion Matrix for Naive Bayes}
\end{table}

\begin{table}[h]
	\center
	\begin{tabular}{ c | c  c  c }
		& \textsc{Precision} & \textsc{Recall} & \textsc{F1-Score} \\
		\hline
		\textsc{$g$} 	& 0.68 & 0.68 & 0.68 \\
		\textsc{$s$}	& 0.68 & 0.68 & 0.68
	\end{tabular}
	\caption{First Tier: Decision Tree P/R/F scores}
\end{table}

\begin{table}[h]
	\center
	\begin{tabular}{ c | c  c }
		 & \textsc{actual $g$} & \textsc{actual $s$} \\
		\hline
		\textsc{predicted $g$} 	& 19 & 9 \\
		\textsc{predicted $s$}		& 9 & 19
	\end{tabular}
	\caption{First Tier: Confusion Matrix for Decision Tree}
\end{table}
\newpage

\begin{table}[h]
	\center
	\begin{tabular}{ c | c  c  c }
		& \textsc{Precision} & \textsc{Recall} & \textsc{F1-Score} \\
		\hline
		\textsc{$g$} 	& 0.91 & 0.71 & 0.80 \\
		\textsc{$s$}	& 0.76 & 0.93 & 0.84
	\end{tabular}
	\caption{Second Tier: SVM P/R/F scores}
\end{table}

\begin{table}[h]
	\center
	\begin{tabular}{ c | c  c }
		 & \textsc{actual $g$} & \textsc{actual $s$} \\
		\hline
		\textsc{predicted $g$} 	& 20 & 8 \\
		\textsc{predicted $s$}		& 2 & 26
	\end{tabular}
	\caption{Second Tier:Confusion Matrix for SVM}
\end{table}

\begin{table}[h]
	\center
	\begin{tabular}{ c | c  c  c }
		& \textsc{Precision} & \textsc{Recall} & \textsc{F1-Score} \\
		\hline
		\textsc{$g$} 	& 0.83 & 0.86 & 0.84 \\
		\textsc{$s$}	& 0.85 & 0.82 & 0.84
	\end{tabular}
	\caption{Second Tier:Naive Bayes P/R/F scores}
\end{table}

\begin{table}[h]
	\center
	\begin{tabular}{ c | c  c }
		 & \textsc{actual $g$} & \textsc{actual $s$} \\
		\hline
		\textsc{predicted $g$} 	& 24 & 4 \\
		\textsc{predicted $s$}		& 5 & 23
	\end{tabular}
	\caption{Second Tier:Confusion Matrix for Naive Bayes}
\end{table}

\begin{table}[h]
	\center
	\begin{tabular}{ c | c  c  c }
		& \textsc{Precision} & \textsc{Recall} & \textsc{F1-Score} \\
		\hline
		\textsc{$g$} 	& 0.81 & 0.79 & 0.80 \\
		\textsc{$s$}	& 0.79 & 0.82 & 0.81
	\end{tabular}
	\caption{Second Tier:Decision Tree P/R/F scores}
\end{table}

\begin{table}[h]
	\center
	\begin{tabular}{ c | c  c }
		 & \textsc{actual $g$} & \textsc{actual $s$} \\
		\hline
		\textsc{predicted $g$} 	& 22 & 6 \\
		\textsc{predicted $s$}		& 5 & 23
	\end{tabular}
	\caption{Second Tier:Confusion Matrix for Decision Tree}
\end{table}
\end{document}
