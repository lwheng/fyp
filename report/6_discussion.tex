\chapter{Discussion}
\label{discussion}
% Min: You need to actually discuss performance of your classifier somewhere here too.  It doesn't make sense to have an evaluation section talking about percentages and a discussion section that doesn't follow up.
Citation Provenance is a task that has had little development up to now. In this thesis, we have rigorously defined the problem and decomposed it into two tiers.  A key limitation in enabling this research is the difficulty in obtaining Specific citations in scientific papers in the ground truth. 
% Min: you need to mention what neutral citations are for.  Some people will just read the conclusion and introduction, so this may not make sense.
\outcite{teufel2009annotation} showed that the percentage of neutral citations was 62.7\%. 
% Min: I really don't understand what you are saying here.  
% Min: Also, did you end up using Eric's classifier as an input?  This sentence seems to be related to that.
We can say that the percentage of General citations is at least as much, because our definition of a Specific citation is more restricted compared to the 12 classes defined by \outcite{teufel2009annotation}. This supports our observations during annotation collection that most citations are mere {\it mentions}.

Sometimes, in-line citations to scientific papers in journals and books capture the chapter numbers and page numbers. The main reason is because the length of the cited document is very long compared to the citing document. An example of such citation is \textit{(J. Doe, 2012, sec. 6.5, 174-85)}. In this citation it captures the section number, ``\textit{sec. 6.5}", and page numbers, ``\textit{174-85}", to a book or journal. Note that the granularity of such style is not specific enough for our problem as a section can be arbitrary lengthy. In our thesis, we have used an archive of computational linguistics papers due to their convenient citation linking data.  However, they are usually less than 20 pages, which is much shorter than books and journals.  While outside of the scope of this thesis, here we sketch a new citation style that may better captures citation provenance.

Our sketched style is straightforward: Authors should numerically label each segment or fragment in the cited paper. This applies to text bodies, figures and tables. An example for a Specific citation: \textit{(B. White, 2011, \textbf{B23})}. Notice we added another a \textbf{B} to \textbf{23}, which could be a better way to distinguish between text bodies (B), figures (F) and tables (T). \textbf{23} simply means the 23rd segment of the type B. Suppose the cited paper is already labelled, when a reader sees a citation a paper, the reader sees there is the additional information at the end of the citation and understands it is a Specific citation. To read up on the cited paper would be a breeze.

We argue that even though the occurrence of Specific citations is low in the examined dataset, citation provenance is an important reading tool to assist in the understanding and navigation between papers linked by citations. 
% Min: not grammatical.  
% Min: you should factor this into a separate subsection, putting in description and screenshots of your CFS application. 
% Min: don't use url for non-urls.  BUG.  Please fix.
We support with evidence the validity of our claim, that a prototype application (that performs Citation Provenance) submitted as part of the \url{Code For Science}\footnote{http://www.codeforscience.com/singapore} 2012 competition organised by Elsevier was well received among the judging panel that consisted of professionals from fields related to information technology and libraries.

