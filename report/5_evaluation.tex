\chapter{Evaluation}
\label{evaluation}
I first performed modular evaluation on {\it GvS} and {\it LocateProv} separately. For each tier I evaluate its performance on a few classifiers: Support Vector Machine (SVM), Naive Bayes (NB) and Decision Tree (DT). For each classifier I also performed evaluation using a few evaluation strategies.

\section{Evaluating {\it GvS}}
Recall that I used a $1:1$ of Specific versus General data instances for building the model. To verify {\it GvS}, I first evaluated the features added using the {\it feature ablation} strategy. For each feature removed from this set of unskewed data instances, the rest of the features are used to train a model using the SVM classifier and then tested on the same set of data instances. To measure the performance each round, I used the conventional accuracy measure. Note that in Figure \ref{fig:ablation_first} the letters $A$ to $E$ represents the five features described in Chapter \ref{firsttier}.

\begin{figure}[ht]
\begin{minipage}[b]{0.45\linewidth}\centering
\begin{tabular}{ l | l }
Configuration & Accuracy \\
\hline
Full		& 0.911 \\
Full - A	& 0.714 \\
Full - B	& 0.875 \\
Full - C	& 0.786 \\
Full - D	& 0.911 \\
Full - E	& 0.732 \\
\end{tabular}
\end{minipage}
\hspace{0.5cm}
\begin{minipage}[b]{0.45\linewidth}\centering
\begin{tabular}{ c | l }
Configuration & Accuracy \\
\hline
Only A	& 0.696 \\
Only B	& 0.589 \\
Only C	& 0.625 \\
Only D	& 0.535 \\
Only E	& 0.696 \\
\end{tabular}
\end{minipage}
\caption{Feature Ablation on {\it GvS}}
\label{fig:ablation_first}
\end{figure}

We observe that feature $A$ (Physical Feature) has the most impact in the accuracy of the predictions, with the greatest drop in accuracy when $A$ itself is removed and one of the highest accuracy when it alone is used as a feature (see Figure \ref{fig:ablation_first}). Feature $D$ (Citing Context's Average \url{tf-idf} Weight) appears to be the only redundant feature, but since it does not decrease the overall accuracy we shall include it nevertheless.

We continue evaluating {\it GvS} using the \url{Leave-One-Out} strategy.

%\begin{table}[h]
%	\center
%	\begin{tabular}{ c | c  c  c }
%		\textsc{Classifier/Values} & \textsc{Avg. Precision} & \textsc{Avg. Recall} & \textsc{Avg. F$_1$-Score} \\
%		& \url{LOO} / \url{n-Fold} & \url{LOO} / \url{n-Fold} & \url{LOO} / \url{n-Fold} \\
%		\hline
%		\textsc{SVM} 			& 0.84 / 0.71 & 0.84 / 0.70 & 0.84 / 0.69 \\
%		\textsc{NaiveBayes} 	& 0.70 / 0.70 & 0.66 / 0.66 & 0.64 / 0.64 \\
%		\textsc{DecisionTree}	& 0.72 / 0.66 & 0.71 / 0.66 & 0.71 / 0.66
%	\end{tabular}
%	\caption{First Tier Results}
%	\label{tab:firsttieresults}
%\end{table}
%
%We examine the confusion matrix for the best performing SVM classifier that we ran for the \url{Leave-One-Out} strategy. Recall that our First Tier's objective is to filter out the General citations. Our goal is to attain higher numbers in both the $g$-$g$ and $s$-$s$ cells in the confusion matrix. We achieved this in Table \ref{tab:svmconfusionmatrix} and we can conclude that our First Tier performed well in differentiating General and Specific citations.
%
%\begin{table}[h]
%	\center
%	\begin{tabular}{ c | c  c }
%		 & \textsc{actual $g$} & \textsc{actual $s$} \\
%		\hline
%		\textsc{predicted $g$} 	& 24 & 4 \\
%		\textsc{predicted $s$}		& 5 & 23
%	\end{tabular}
%	\caption{Confusion Matrix for SVM with Leave-One-Out}
%	\label{tab:svmconfusionmatrix}
%\end{table}
%
%\section{Results - Second Tier}
%For Second Tier evaluation, we are predicting whether each fragment in the cited paper is a Specific one. We have over 30 thousand training instances for second tier, and so for the same reason, we had to select our training set manually similarly to get a $1:1$ ratio for Specific versus General instances.
%
%\begin{table}[h]
%	\center
%	\begin{tabular}{ c | c  c  c }
%		\textsc{Classifier/Values} & \textsc{Avg. Precision} & \textsc{Avg. Recall} & \textsc{Avg. F$_1$-Score} \\
%		& \url{LOO} / \url{n-Fold} & \url{LOO} / \url{n-Fold} & \url{LOO} / \url{n-Fold} \\
%		\hline
%		\textsc{SVM} 			& 0.85 / 0.84 & 0.84 / 0.82 & 0.84 / 0.82 \\
%		\textsc{NaiveBayes} 	& 0.80 / 0.78 & 0.79 / 0.77 & 0.78 / 0.77 \\
%		\textsc{DecisionTree}	& 0.89 / 0.86 & 0.89 / 0.86 & 0.89 / 0.86
%	\end{tabular}
%	\caption{Second Tier Results}
%	\label{tab:secondtieresults}
%\end{table}
%\newpage
%In this case, the Decision Tree classifier performed slightly better than SVM. Similarly, our goal is to attain higher numbers in both the $g$-$g$ and $s$-$s$ cells in the confusion matrix and in Table \ref{tab:decisiontreeconfusionmatrix} we achieved good results. This means, given a Specific citation, this approach would perform well in determining whether a fragment in the cited paper is the cited fragment or not.
%
%\begin{table}[h]
%	\center
%	\begin{tabular}{ c | c  c }
%		 & \textsc{actual $g$} & \textsc{actual $s$} \\
%		\hline
%		\textsc{predicted $g$} 	& 23 & 5 \\
%		\textsc{predicted $s$}		& 3 & 25
%	\end{tabular}
%	\caption{Confusion Matrix for Naive Bayes}
%	\label{tab:decisiontreeconfusionmatrix}
%\end{table}
%
%(Refer to Appendix \ref{resultsdetails} for more details of our experimental results.)