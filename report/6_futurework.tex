\chapter{Discussion}
\label{discussion}
\paragraph{}
Citation Provenance is a task that has relatively little developments done on it. In our paper we attempt to define the nature of the problem, and presented a possible approach to tackle it. One of the main challenges we had with this task is the limited number of Specific citations in scientific papers. \cite{teufel2009annotation} showed that the percentage of neutral citations was 62.7\%. We can say that the percentage of General citations is at least as much, because our definition of a Specific citation is much more restricted compared to the 12 classes defined in \cite{teufel2009annotation}. We observed, and conclude that in general research \& scientific papers, most citations are neutral and General.

\paragraph{}
We argue, that even though the percentage of Specific citations is very low and that the value of applications that perform such task seems low, citation provenance would prove to be an important reading tool that helps readers understand and navigate between papers that are linked via citations. We support with evidence the validity of our claim, that our prototype application submitted as part of the \url{Code For Science}\footnote{http://www.codeforscience.com/singapore} 2012 competition organised by Elsevier was well received among the judging panel that consisted of professionals from fields related to information technology and libraries.

%Tao: Include the actual link (http://www.codeforscience.com/singapore) of codeforscience competition 