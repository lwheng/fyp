\chapter{Discussion}
\label{discussion}
\paragraph{}
Citation Provenance is a task that has relatively little developments done on it. In our paper we attempt to define the nature of the problem, and presented a possible approach to tackle it. One of the main challenges we had with this task is the limited number of Specific citations in scientific papers. \cite{teufel2009annotation} showed that the percentage of neutral citations was 62.7\%. We can say that the percentage of General citations is at least as much, because our definition of a Specific citation is much more restricted compared to the 12 classes defined in \cite{teufel2009annotation}. We observed, and conclude that in general research \& scientific papers, most citations are neutral and General.

\paragraph{}
We argue, that even though the percentage of Specific citations is low and that the value of applications that perform such task seems low, citation provenance would prove to be an important reading tool that helps readers understand and navigate between papers that are linked via citations. We support with evidence the validity of our claim, that our prototype application submitted as part of the \url{Code For Science}\footnote{http://www.codeforscience.com/singapore} 2012 competition organised by Elsevier was well received among the judging panel that consisted of professionals from fields related to information technology and libraries.

\paragraph{}
Sometimes, in-line citations to scientific papers in journals and books capture the chapter numbers and page numbers. The main reason is because the length of the cited document is very long compared to the citing document. An example of such citation is \textit{(J. Doe, 2012, sec. 6.5, 174-85)}. In this citation it captures the section number, ``\textit{sec. 6.5}", and page numbers, ``\textit{174-85}", to a book or journal. Note that the granularity of such style is not specific enough for our problem as a section can be arbitrary lengthy. In our case, we consider computational linguistic papers that are usually less than 20 pages, which is much shorter than books and journals. For this we sketch a new citation style that better captures citation provenance.

\paragraph{}
Our sketched style is straightforward: To numerically label each segment or fragment in the cited paper. This applies to text bodies, figures and tables. An example for a Specific citation: \textit{(B. White, 2011, \textbf{B23})}. Notice we added another a \textbf{B} to \textbf{23}, which could be a better way to distinguish between text bodies (B), figures (F) and tables (T). \textbf{23} simply means the 23rd segment of the type B. Suppose the cited paper is already labelled, when a reader sees a citation a paper, the reader sees there is the additional information at the end of the citation and understands it is a Specific citation. To read up on the cited paper would be a breeze.