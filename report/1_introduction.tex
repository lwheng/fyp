\chapter{Introduction}
\label{introduction}
Citing previously published scientific papers is an important practice among researchers. It gives credit and acknowledgement to original ideas and to researchers who did significant work in enabling the current research.  More importantly, it upholds intellectual property. A reader of such research papers often encounters these citations made by the authors in various sentences throughout the paper. When a reader wishes to gain a better understanding of the current context, it is necessary to follow these citations and read the cited papers to understand the basis for the current work.  Often, when reading the claims of a sentence supported by a citation, readers wish to know where in the cited paper the information comes from.  

However, as frequent readers might find, most citations are only \textit{mentions}. They do not directly refer to some particular section of the cited paper, for example, to make reference to the evaluation results made by the authors of the cited paper. Instead, they are what we term {\it general} citations. Other citations refer specifically to particular claims, parts or sections of a paper.  These citations are equally important.  However, since it may not be immediately clear where the cited information is from\footnote{page numbers or references to specific artifacts, such as sections or equation numbers sometimes help to localize such references, but are not often included.}, a reader has to invest additional effort to locate the cited information. We shall refer to \cite{citation-sensitive} for their survey results to justify my claims. In the series of surveys they conducted, most of their participants found it difficult \textit{finding the exact text to justify the citation}. Quoting one of their participants' response directly: ``\textit{Citation usually does not include the position of the information} in the cited article... it might be necessary to read all of the article to find it in another reference and so on." \cite{citation-sensitive}

{\it Citation Provenance} refers to the source of a citation. The task of determining citation provenance is to locate the information in the cited paper that justifies the citation. It improves the reading experience of scientific and research documents by showing where exactly the cited information is from in the cited paper. We aim to identify which section or paragraph in the referenced paper is the cited information. 

In comparison with \cite{csibs}, which only provided a summarisation solution, this paper describes the first attempt to provide a solution to the difficulty with locating the information that justifies a citation. We hope this would also encourage meaningful discussions to designing a new citation style that better captures the provenance of the cited information.

In the rest of this paper, we will first look at some related works. In Chapter \ref{problemanalysis}, We analyse the problem and describe some observations made from building the corpus. In Chapter \ref{approach} we discuss my approach on tackling the problem. We present our experimental results in Chapter \ref{evaluation} followed by the conclusion.
