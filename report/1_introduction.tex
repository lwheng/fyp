\chapter{Introduction}
\label{introduction}
% Min: don't need these
% \paragraph{}
Citing previously published scientific papers is an important practice among researchers. It gives credit and acknowledgement to original ideas, and to researchers who did significant work in enabling the current research.  More importantly, it upholds intellectual property. A reader of such research papers often encounters these citations made by the authors in various sentences throughout the paper. Often enough, if a reader wishes to gain a better understanding of the current context, it is necessary to follow these citations and read the cited papers to understand the basis for the current work.  Often, when reading the claims of a sentence supported by a citation, readers wish to know where in the cited paper the information comes from.  

% Min: prefer I and me for thesis.  It's singly authored work.
However, as frequent readers might find, most citations are only \textit{mentions}. They do not directly refer to some particular section of the cited paper, for example, to make reference to the evaluation results made by the authors of the cited paper. Instead, they are what I term {\it general} citations. Other citations refer specifically to particular claims, parts or sections of a paper.  These citations are equally important.  However, since it may not be immediately clear where the cited information is from\footnote{page numbers or references to specific artifacts, such as sections or equation numbers sometimes help to localize such references, but are not often included.}, a reader has to invest additional effort to locate the cited information. We refer to \cite{citation-sensitive} for their survey results to justify our claims. In the series of surveys they conducted, most of their participants found it difficult \textit{finding the exact text to justify the citation}. We quote one of their participants' response directly: ``\textit{Citation usually does not include the position of the information} in the cited article... it might be necessary to read all of the article to find it in another reference and so on." \cite{citation-sensitive}

% Min: Thu Nov  1 09:57:41 SGT 2012 Stopped here
% \paragraph{}
In general, we wish to improve the reading experience of scientific and research documents, from the various fields of research. Readers will be informed of where exactly the cited information is from in the cited paper. We aim to be able to identify which section or paragraph in the referenced paper is the source of the information referred to in the citation, i.e. Citation Provenance. In comparison with \cite{csibs}, which only provided a summarisation solution, we are the first to provide a solution to the difficulty in locating the information that justifies a citation. After highlighting this problem, we hope this would encourage meaning discussions to designing a new citation style that better captures the provenance of the cited information.
 
% \paragraph{}
In the rest of this paper, we will first look at some past works that are related to what we are describing. In Chapter \ref{approach}, we analyse the problem, and describe our approach on tackling the problem. We present our experimental results in Chapter \ref{evaluation} and finally in the end, we conclude our paper together with some further discussion.
