\chapter{Introduction}
\label{introduction}
\paragraph{}
Citing previously published scientific papers is a necessary and common practice among researchers. It gives credit and acknowledgement to original ideas, and to researchers who did significant work in a particular field of research, and more importantly, upholds intellectual property. A reader of such research papers often encounters these citations made by the authors in various sentences throughout the paper. Often enough, if a reader wishes to gain a better understanding of the current context, it is necessary to follow through these citations and read up on these cited papers. Readers would also be interested to know where the information is in the cited paper.

\paragraph{}
However, as frequent readers might find, most citations are only \textit{mentions}. They do not directly refer to some particular section of the cited paper, for example, to make reference to the evaluation results made by the authors of the cited paper. Instead, they are general citations. These citations are equally important. But since it is not immediately clear where the cited information is from, a reader has to invest additional time to read through the entire cited paper before being able to find out what or where the critical information is. We refer to \cite{citation-sensitive} for their survey results to justify our claims. In the series of surveys they conducted, most of their participants found it difficult \textit{finding the exact text to justify the citation}. We quote one of their participants' response directly: ``\textit{Citation usually does not include the position of the information} in the cited article... it might be necessary to read all of the article to find it in another reference and so on." \cite{citation-sensitive}

\paragraph{}
In general, we wish to improve the reading experience of scientific and research documents, from the various fields of research. Readers will be informed of where exactly the cited information is from in the cited paper. We aim to be able to identify which section or paragraph in the referenced paper is the source of the information referred to in the citation, i.e. Citation Provenance. In comparison with the continued work done by \cite{csibs}, which provided a summarisation solution, our goal is to provide the \textit{answer} to a reader's doubts when a citation appears in a paper. Also, possibly, after introducing this new problem, encourage the increase of meaningful and contributive citations in scientific papers.

%Tao: You need to state the novelty of your work. E.g., the  second work that tackle this problem, and the advantage of your work than Wan's work.
 
\paragraph{}
In the rest of this paper, we will first look at some past works that are related to what we are describing. In Chapter \ref{approach}, we analyse the problem, and describe our approach on tackling the problem. We present our experimental results in Chapter \ref{evaluation} and finally in the end, we conclude our paper together with some further discussion.